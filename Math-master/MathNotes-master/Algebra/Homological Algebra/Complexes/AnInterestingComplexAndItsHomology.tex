%% LyX 2.2.3 created this file.  For more info, see http://www.lyx.org/.
%% Do not edit unless you really know what you are doing.
\documentclass[12pt,english]{article}
\usepackage[osf]{mathpazo}
\renewcommand{\sfdefault}{lmss}
\renewcommand{\ttdefault}{lmtt}
\usepackage[T1]{fontenc}
\usepackage[latin9]{inputenc}
\usepackage[paperwidth=30cm,paperheight=35cm]{geometry}
\geometry{verbose,tmargin=3cm,bmargin=3cm}
\setlength{\parindent}{0bp}
\usepackage{amsmath}
\usepackage{amssymb}

\makeatletter

%%%%%%%%%%%%%%%%%%%%%%%%%%%%%% LyX specific LaTeX commands.
%% Because html converters don't know tabularnewline
\providecommand{\tabularnewline}{\\}

%%%%%%%%%%%%%%%%%%%%%%%%%%%%%% User specified LaTeX commands.
\usepackage{tikz}
\usetikzlibrary{matrix,arrows,decorations.pathmorphing}
\usetikzlibrary{shapes.geometric}
\usepackage{tikz-cd}
\usepackage{amsthm}
\theoremstyle{plain}
\newtheorem{theorem}{Theorem}[section]
\newtheorem{lemma}[theorem]{Lemma}
\newtheorem{prop}{Proposition}[section]
\newtheorem*{cor}{Corollary}
\theoremstyle{definition}
\newtheorem{defn}{Definition}[section]
\newtheorem{ex}{Exercise} 
\newtheorem{example}{Example}[section]
\theoremstyle{remark}
\newtheorem*{rem}{Remark}
\newtheorem*{note}{Note}
\newtheorem{case}{Case}
\usepackage{graphicx}
\usepackage{amssymb}
\usepackage{tikz-cd}
\usetikzlibrary{calc,arrows,decorations.pathreplacing}
\tikzset{mydot/.style={circle,fill,inner sep=1.5pt},
commutative diagrams/.cd,
  arrow style=tikz,
  diagrams={>=latex},
}

\usepackage{babel}
\usepackage{hyperref}
\hypersetup{
    colorlinks,
    citecolor=black,
    filecolor=black,
    linkcolor=black,
    urlcolor=black
}
\usepackage{pgfplots}
\usetikzlibrary{decorations.markings}
\pgfplotsset{compat=1.9}

\makeatother

\usepackage{babel}
\begin{document}

\title{An Interesting Complex and its Homology}

\author{Michael Nelson}

\maketitle
\pagebreak{}

\tableofcontents{}

\pagebreak{}

\section{Notations and Preliminary Material}

~~~Throughout this article, let $R$ be a ring. Recall that a \textbf{chain
complex} $A=(A_{\bullet},d_{\bullet})$ \textbf{over} $R$ is a sequence
of $R$-modules $A_{i}$ and morphisms $d_{i}:A_{i}\to A_{i-1}$ \begin{equation}\label{diagramA}\begin{tikzcd} A := \cdots \arrow[r] & A_{i+1} \arrow[r,"d _{i+1}"] & A_i  \arrow[r," d _i "] & A_{i-1} \arrow[r] & \cdots \end{tikzcd}\end{equation}

such that $d_{i}\circ d_{i+1}=0$ for all $i\in\mathbb{Z}$. The condition
$d_{i}\circ d_{i+1}=0$ is equivalent to the condition $\text{Ker}(d_{i})\supset\text{Im}(d_{i+1})$.
With this in mind, we define the \textbf{$i$th} \textbf{homology
of the chain complex} to be 
\[
H_{i}(A):=\text{Ker}(d_{i})/\text{Im}(d_{i+1}).
\]
A \textbf{chain map} between two chain complexes $(A_{\bullet},d_{\bullet})$
and $(A'_{\bullet},d'_{\bullet})$ over $R$ is a sequence $\varphi_{\bullet}$
of $R$-module homomorphisms $\varphi_{i}:A_{i}\to B_{i}$ such that
$d_{i}\varphi_{i-1}=\varphi_{i}d_{i-1}'$ for all $i$. It then follows
that a chain map gives rise to map an induced map on homology $H_{i}(A)\to H_{i}(A')$. 

~~~~~~To simplify notation, we think of $R$ as a trivially
graded ring, that is, the degree equals $0$ part is $R$ and all
the other homogeneous components are $0$. We think of the complex
(\ref{diagramA}) as a graded $R$-module $A$ (where the degree $i$
homogoneous component is $A_{i}$) together with an endomorphism $d$
of degree $-1$ such that $d^{2}=0$. We write $A[j]$ for the graded
module obtained from $A$ by the rule $A[j]_{i}=A_{i+j}$. 

\section{Introducing $\mathbf{A}(I),$ $\mathbf{A}(S)$, and $\mathbf{A}(S\backslash I)$}

~~~Let $S=\mathbb{F}_{2}[x_{1},\dots,x_{n}]$ and let $I$ be a
monomial ideal in the polynomial ring $\mathbb{F}_{2}[x_{1},\dots,x_{n}]$.
For $i\geq0$, let $S_{i}$ denote the $\mathbb{F}_{2}$-vector space
generated by the monomials $m\in\mathbb{F}_{2}[x_{1},\dots,x_{n}]$
of degree $i$ such that $m\notin I$, and let $d_{i}:=\sum_{j=1}^{n}\partial_{x_{j}}$.
If $i<0$, we simply set $S_{i}=0$. Then we have the following chain
complexes over $\mathbb{F}_{2}$:
\begin{center}
\begin{tabular}{c|c|c}
Chain Complex & Homogeneous Components & Differential\tabularnewline
\hline 
\hline 
$\mathbf{A}(S):=(S_{\bullet},d_{\bullet})$ & $\mathbf{A}(S)_{i}:=S_{i}$  & $d_{i}:S_{i}\to S_{i-1}$\tabularnewline
\hline 
$\mathbf{A}(I):=((S\cap I)_{\bullet},d_{\bullet})$ & $\mathbf{A}(I)_{i}:=S_{i}\cap I$ & $\overline{d}_{i}:S_{i}\cap I\to S_{i-1}\cap I$\tabularnewline
\hline 
$\mathbf{A}(S\backslash I):=((S\backslash I)_{\bullet},d_{\bullet})$ & $\mathbf{A}(S\backslash I)_{i}:=S_{i}\backslash I$ & $d_{i}:S_{i}\backslash I\to S_{i-1}\backslash I$\tabularnewline
\end{tabular}
\par\end{center}

where $\overline{d}_{i}(f)=d_{i}(f)\text{ mod }(S_{i-1}\backslash I)$
and $d_{i}:S_{i}\backslash I\to S_{i-1}\backslash I$ is understood
to just be the restriction of $d_{i}$ to $S_{i}\backslash I$. To
ease notation, we drop the subscript $i$ in $d_{i}$ whenever the
context is clear (which is usually the case). To see that these really
are chain complexes over $\mathbb{F}_{2}$, note that every homogeneous
component is an $\mathbb{F}_{2}$-vector space, and the differential
is $\mathbb{F}_{2}$-linear. To see that $d^{2}=0$, we just need
to check that $d^{2}\left(x_{1}^{\alpha_{1}}\cdots x_{n}^{\alpha_{n}}\right)=0$
for any monomial $x_{1}^{\alpha_{1}}\cdots x_{n}^{\alpha_{n}}$ in
$\mathbb{F}_{2}[x_{1},\dots,x_{n}]$. This follows since we are working
mod $2$:
\begin{align*}
d^{2}\left(x_{1}^{\alpha_{1}}\cdots x_{n}^{\alpha_{n}}\right) & =\left(\sum_{k=1}^{n}\partial_{x_{k}}\right)^{2}\left(x_{1}^{\alpha_{1}}\cdots x_{n}^{\alpha_{n}}\right)\\
 & =\left(\sum_{k=1}^{n}\partial_{x_{k}}^{2}\right)\left(x_{1}^{\alpha_{1}}\cdots x_{n}^{\alpha_{n}}\right)\\
 & =\sum_{k=1}^{\infty}\alpha_{k}(\alpha_{k}-1)x_{1}^{\alpha_{k}-2}\\
 & =0.
\end{align*}

\begin{rem}\label{rem}Since $\mathbf{A}(S\backslash I)$ is a chain
complex over $\mathbb{F}_{2}$, we can think of $\mathbf{A}(I)$ as
a graded $\mathbb{F}_{2}$-module together with an $\mathbb{F}_{2}$-endomorphism
$d$ of degree $-1$ such that $d^{2}=0$. We can also realize $\mathbf{A}(I)$
as a graded $S$-module, namely $S/I$. However, $d$ is not an $S$-endomorphism,
so we cannot think of it as a chain complex over $S$. \end{rem}

\begin{example}\label{example} To get an idea of how this construction
looks, consider the case where $S=\mathbb{F}_{2}[x,y,z]$ and $I=\langle xy,x^{3},y^{2}z,xz^{3},z^{4},y^{5}\rangle$.
Let's write down the graded pieces of $\mathbf{A}(S\backslash I)$:
\begin{align*}
 & =\vdots\\
\mathbf{A}(S\backslash I)_{-1} & =0\\
\mathbf{A}(S\backslash I)_{0} & =\mathbb{F}_{2}\\
\mathbf{A}(S\backslash I)_{1} & =\mathbb{F}_{2}x+\mathbb{F}_{2}y+\mathbb{F}_{2}x\\
\mathbf{A}(S\backslash I)_{2} & =\mathbb{F}_{2}x^{2}+\mathbb{F}_{2}xz+\mathbb{F}_{2}y^{2}+\mathbb{F}_{2}yz+\mathbb{F}_{2}z^{2}\\
\mathbf{A}(S\backslash I)_{3} & =\mathbb{F}_{2}x^{2}z+\mathbb{F}_{2}xz^{2}+\mathbb{F}_{2}y^{3}+\mathbb{F}_{2}yz^{2}+\mathbb{F}_{2}z^{3}\\
\mathbf{A}(S\backslash I)_{4} & =\mathbb{F}_{2}x^{2}z^{2}+\mathbb{F}_{2}y^{4}+\mathbb{F}_{2}yz^{3}\\
\mathbf{A}(S\backslash I)_{5} & =0\\
 & \vdots
\end{align*}
Since $I$ is a monomial ideal, there exists a unique minimal set
of generators $G$. In this case, $G=\{xy,x^{3},y^{2}z,xz^{3},z^{4},y^{5}\}$.
We can get an idea of how the first few graded pieces of $\mathbf{A}(I)$
look using $G$
\begin{align*}
 & =\vdots\\
\mathbf{A}(I)_{1} & =0\\
\mathbf{A}(I)_{2} & =\mathbb{F}_{2}xy\\
\mathbf{A}(S\backslash I)_{3} & =\mathbb{F}_{2}x^{3}+\mathbb{F}_{2}x^{2}y+\mathbb{F}_{2}xy^{2}+\mathbb{F}_{2}xyz+\mathbb{F}_{2}y^{2}z\\
 & \vdots
\end{align*}

Note that $\overline{d}(xy)=\overline{d(xy)}=\overline{x+y}=\overline{0}$
since $x+y\in S\backslash I$. \end{example}

\subsection{Differential Graded Algebra Structure}

~~~The chain complex $\mathbf{A}(S)$ is more than just a chain
complex, in fact it has the structure of a differential graded algebra
over $\mathbb{F}_{2}$. A \textbf{differential graded algebra over
$R$ }is a chain complex $A=(A_{\bullet},d_{\bullet})$ over $R$
together with $R$-bilinear maps $A_{i}\times A_{j}\to A_{i+j}$,
denoted $(a,b)\mapsto ab$, such that the \textbf{Leibniz law }holds:
\[
d_{i+j}(ab)=d_{i}(a)b+(-1)^{i}ad_{j}(b).
\]
for all $a,b\in A$. In this case, the $\mathbb{F}_{2}$-bilinear
maps are just multiplication.

~~~On the other hand, if $I\neq0$, then $\mathbf{A}(I)$ is not
differential graded algebra over $\mathbb{F}_{2}$: The $\mathbb{F}_{2}$-bilinear
maps in this case are multipliation, however we do not get Leibniz
law in this case. It is true, that for most pairs $(f,g)\in I^{2}$,
the Leibniz law will be satisfied: 
\[
d(fg)=d(f)g+fd(g).
\]
However, there may exist a pair for which the Leibniz law is not satisfied.
For instance, consider $I=\langle x^{5}\rangle$ in $\mathbb{F}_{2}[x]$.
Then 
\[
x^{10}=\overline{d}(x^{5}\cdot x^{6})=\overline{d}(x^{5})x^{6}+x^{5}\overline{d}(x^{6})=0
\]
leads to a contradiction. The problem here is that $\overline{d}(x^{5})=0$
and $d(x^{6})=0$. We shall see that existence of this of pair is
due to the existence of a nontrivial element in homology. In particular,
the pair $(x^{5},x^{6})$ corresponds to the element $\overline{x}^{5}\in H_{5}(\mathbf{A}(I))$. 

~~~Similarly, $\mathbf{A}(S\backslash I)$ is not a differential
graded algebra over $\mathbb{F}_{2}$: The $\mathbb{F}_{2}$-bilinear
maps in this case are multipliation mod $I$, however we do not get
Leibniz law in this case. For instance, consider $I=\langle x^{5}\rangle$
in $\mathbb{F}_{2}[x]$ again. Then 
\[
0=d(\overline{x}\cdot\overline{x}^{4})=\overline{d(x)}\overline{x}^{4}+\overline{x}\overline{d(x^{4})}=\overline{x}^{4}
\]
leads to a contradiction. The problem this time is that $\overline{x}^{5}=0$
and $d(x^{4})=0$. As in the case of $\mathbf{A}(I)$, we shall see
that existence of this of pair is due to the existence of a nontrivial
element in homology. In particular, the pair $(x,x^{4})$ corresponds
to the element $\overline{x}^{4}\in H_{4}(\mathbf{A}(S\backslash I))$. 

\subsection{Calculating $H_{i}(\mathbf{A}(I)),$ $H_{i}(\mathbf{A}(S))$, and
$H_{i}(\mathbf{A}(S\backslash I))$}

\begin{prop}\label{prop} $H_{i}(\mathbf{A}(S))=0$ for all $i\geq0$.
\end{prop}

\begin{rem}\label{rem} The proof of this proposition illustrates
the usefulness of the Leibniz law. \end{rem}

\begin{proof} Let $f$ be a homogeneous polynomial of degree $i$
such that $d(f)=0$. Then for any $x_{j}$, we have
\[
d(x_{j}f)=d(x_{j})f+x_{j}d(f)=f.
\]
Therefore, $\text{Ker}(d)=\text{Im}(d)$, which proves the claim.
\end{proof}

\begin{prop}\label{prop} The differential $d$ induces isomorphisms
$H_{i}(\mathbf{A}(I))\cong H_{i-1}(\mathbf{A}(S\backslash I))$ for
all $i>0$. \end{prop}

\begin{proof} From the way we constructed $\mathbf{A}(I)$, $\mathbf{A}(S)$,
and $\mathbf{A}(S\backslash I)$, we have a short exact sequence of
chain complexes

\begin{center}\begin{tikzcd} 0 \arrow[r] & \mathbf{A}(S \setminus I) \arrow[r, hook] & \mathbf{A}(S) \arrow[r] & \mathbf{A}(I) \arrow[r] & 0. \end{tikzcd}\end{center}

From this short exact sequence, we obtain, for all $i>0$, the exact
sequences

\begin{center}\begin{tikzcd} 0 = H_i ( \mathbf{A}(S) ) \arrow[r] & H_i ( \mathbf{A}(I) ) \arrow[r, "d " ] & H_{i-1} ( \mathbf{A}(S \setminus I)) \arrow[r] &  H_{i-1} ( \mathbf{A}(S) ) = 0 \end{tikzcd}\end{center}where
$d$ is obtained by working out the details of the connecting map.

\end{proof}

\begin{prop}\label{prop} For all $i>1$, $H_{2i}(\mathbf{A}(S\backslash I))$
is generated by monomials $m=x_{1}^{\alpha_{1}}\cdots x_{n}^{\alpha_{n}}$
of degree $2i$ such that $\alpha_{j}\in2\mathbb{Z}_{\geq0}$ for
all $1\leq j\leq n$ and $m\in\text{Soc}(S/I)$. \end{prop}

\subsection{The Codifferential $\delta$}

~~~We briefly mention here that one should be able to construct
a codifferential $\delta:S_{i}\to S_{i+1}$ which plays a dual role
to $d$. In particular, the codifferential $\delta$ should induce
isomorphisms $H_{i}(\mathbf{A}(S\backslash I))\cong H_{i-1}(\mathbf{A}(I))$
for all $i>0$. 

\section{$p$-Chain Complexes}

\subsection{$3$-Chain Complexes}

~~~Our goal in this section is to try to generalize the previous
section by replacing the underlying field $\mathbb{F}_{2}$ with the
field $\mathbb{F}_{3}$. Unfortunately in this case, we do not have
$d^{2}\neq0$, so we don't have a chain complex. On the other hand,
we do have $d^{3}=0$. This leads us to the following definition

\begin{defn} A $3$\textbf{-chain complex $A:=(A_{\bullet},d_{\bullet})$
}over $R$ is a sequence of $R$-modules $A_{i}$ and morphisms $d_{i}:A_{i}\to A_{i-1}$

\begin{equation}\label{diagramB}\begin{tikzcd} A := \cdots \arrow[r] & A_{i+1} \arrow[r,"d _{i+1}"] & A_i  \arrow[r," d _i "] & A_{i-1} \arrow[r] & \cdots \end{tikzcd}\end{equation}

such that $d_{i-1}\circ d_{i}\circ d_{i+1}=0$ for all $i\in\mathbb{Z}$.
\end{defn}

\begin{rem}\label{rem} Again, we often drop the subscript $i$ in
$d_{i}$ to ease notation. \end{rem} 

\subsection{Turning a $3$-Chain Complex into a Chain Complex}

~~~In this section, we desribe how we can obtain a chain complex
from a $3$-complex. Let $A:=(A_{\bullet},d_{\bullet})$ be a $3$-chain
complex. 

\begin{center}\begin{tikzcd} A := \cdots \arrow[r] & A_{2} \arrow[r,"d_{2}"] &  A_{1} \arrow[r,"d_{1}"] & A_{0} \arrow[r,"d_{0}"] & A_{-1}  \arrow[r," d_{-1} "] & A_{-2} \arrow[r] & \cdots \end{tikzcd}\end{center}

We \textbf{collapse} $A$ into a chain complex $A_{\star}$ as follows:

\begin{center}\begin{tikzcd} A_{\star } := \cdots \arrow[r] A_{5} \arrow[r,"d_4 d_5 "]  & A_{3} \arrow[r,"d_3 "]  & A_{2} \arrow[r,"d_1 d_2 "]  & A_{0} \arrow[r," d_0 "] & A_{-1}  \arrow[r," d_{-2} d _{-1} "] & A_{-3} \arrow[r, "d_{-3}"] & A_{-4} \arrow[r] & \cdots. \end{tikzcd}\end{center}

More formally, $A_{\star}=(\widetilde{A}_{\bullet},\widetilde{d}_{\bullet})$,
where
\[
\widetilde{A}_{i}=A_{\frac{6i+1+(-1)^{i+1}}{4}}\qquad\widetilde{d}_{i}=\begin{cases}
d_{\frac{6i+1+(-1)^{i+1}}{4}} & \text{if }|i|\text{ is even}\\
d_{\frac{6i-3+(-1)^{i+1}}{4}}d_{\frac{6i+1+(-1)^{i+1}}{4}} & \text{if }|i|\text{ is odd}
\end{cases}
\]

~~~For all $j\in\mathbb{\mathbb{Z}}$, we define $A[j]=(A[j]_{\bullet},d[i]_{\bullet})$
to be the sequence of $R$-modules $A[j]_{i}$ are morphisms $d[j]_{i+j}$
where $A[j]_{i}=A_{i+j}$ and $d[j]=d_{i+j}$. It is straightforward
to verify that $A[j]$ is also a $3$-Chain Complex. We can also define
$A[1]_{\star}$ and $A[2]_{\star}$ in a similar way as $A_{\star}$: 

\begin{center}\begin{tikzcd} 


A_{\star } := \cdots \arrow[r] A_{5} \arrow[r,"d_4 d_5 "]  & A_{3} \arrow[r,"d_3 "]  & A_{2} \arrow[r,"d_1 d_2 "]  & A_{0} \arrow[r," d_0 "] & A_{-1}  \arrow[r," d_{-2} d _{-1} "] & A_{-3} \arrow[r, "d_{-3}"] & A_{-4} \arrow[r] & \cdots

\\

A[1]_{\star } := \cdots \arrow[r] A_{6} \arrow[r,"d_5 d_6 "]  & A_{4} \arrow[r,"d_4 "]  & A_{3} \arrow[r,"d_2 d_3 "]  & A_{1} \arrow[r," d_1 "] & A_{0}  \arrow[r," d_{-1} d _{0} "] & A_{-2} \arrow[r, "d_{-2}"] & A_{-3} \arrow[r] & \cdots

\\

A[2] _{\star } := \cdots \arrow[r] A_{7} \arrow[r,"d_6 d_5 "]  & A_{5} \arrow[r,"d_5 "]  & A_{4} \arrow[r,"d_3 d_4 "]  & A_{2} \arrow[r," d_2 "] & A_{1}  \arrow[r," d_{0} d _{1} "] & A_{-1} \arrow[r, "d_{-1}"] & A_{-2} \arrow[r] & \cdots 


\end{tikzcd}\end{center}

\begin{theorem}\label{theorem} With the notation above, there is
a long exact sequence in homology of the form

\begin{center}\begin{tikzcd}[row sep=30] 

H_{i-1} (A_{\star } ) \arrow[r]  & \cdots \arrow[d, phantom, ""{coordinate, name=Z}] &  & 

\\

H_i (A_{\star } ) \arrow[r, "d_{i+1} "]  & H_{i-1} (A[1]_{\star } ) \arrow[d, phantom, ""{coordinate, name=Z'}] \arrow[r] &  H_{i-2} (A[2]_{\star } ) \arrow[ull, "", swap, rounded corners, to path={ -- ([xshift=2ex]\tikztostart.east) |- (Z) [near end]\tikztonodes -| ([xshift=-2ex]\tikztotarget.west) -- (\tikztotarget)}] 

\\

H_{i+1} (A_{\star } ) \arrow[r]  & H_{i} (A[1]_{\star } ) \arrow[d, phantom, ""{coordinate, name=Z''}] \arrow[r, "d_{i+2} "] &  H_{i-1} (A[2]_{\star } ) \arrow[ull, "", swap, rounded corners, to path={ -- ([xshift=2ex]\tikztostart.east) |- (Z') [near end]\tikztonodes -| ([xshift=-2ex]\tikztotarget.west) -- (\tikztotarget)}] 

\\

H_{i+2} (A_{\star } ) \arrow[r, "d_{i+4} "]  & H_{i+1} (A[1]_{\star } ) \arrow[d, phantom, ""{coordinate, name=Z'''}] \arrow[r] &  H_i (A[2]_{\star } ) \arrow[ull, "d_{i+3} ", swap, rounded corners, to path={ -- ([xshift=2ex]\tikztostart.east) |- (Z'') [near end]\tikztonodes -| ([xshift=-2ex]\tikztotarget.west) -- (\tikztotarget)}] 

\\

& \cdots \arrow[r] & H_{i+1} (A[2]_{\star } ) \arrow[ull, "", swap, rounded corners, to path={ -- ([xshift=2ex]\tikztostart.east) |- (Z''') [near end]\tikztonodes -| ([xshift=-2ex]\tikztotarget.west) -- (\tikztotarget)}] 
 
\end{tikzcd}\end{center}

\end{theorem}

\begin{proof} Let $K_{i}=\text{Ker}(d_{i})/\text{Ker}(d_{i})\cap\text{Im}(d_{i+1})$
for all $i\in\mathbb{Z}$. For each $i\in3\mathbb{Z}$, we have the
following short exact sequences

\begin{center}\begin{tikzcd} 0 \arrow[r] & K_{i+1} \arrow[r]  & H_i (A _{\star }) \arrow[r, "d_{i+1}"]  & H_{i-1} (A[1] _{\star })  \arrow[r]  & K_{i} \arrow[r] & 0

\\

0 \arrow[r] & K_{i+2} \arrow[r]  & H_i (A[1] _{\star }) \arrow[r, "d_{i+2}"]  & H_{i-1} (A[2] _{\star })  \arrow[r]  & K_{i+1} \arrow[r] & 0

\\

0 \arrow[r] & K_{i+3} \arrow[r]  & H_i (A[2] _{\star }) \arrow[r, "d_{i+3}"]  & H_{i-1} (A _{\star })  \arrow[r]  & K_{i+2} \arrow[r] & 0

\end{tikzcd}\end{center}

Connecting these together gives us our desired result. 

\end{proof}
\end{document}
