%% LyX 2.2.3 created this file.  For more info, see http://www.lyx.org/.
%% Do not edit unless you really know what you are doing.
\documentclass[12pt,english]{article}
\usepackage[osf]{mathpazo}
\renewcommand{\sfdefault}{lmss}
\renewcommand{\ttdefault}{lmtt}
\usepackage[T1]{fontenc}
\usepackage[latin9]{inputenc}
\usepackage[paperwidth=30cm,paperheight=35cm]{geometry}
\geometry{verbose,tmargin=3cm,bmargin=3cm}
\setlength{\parindent}{0bp}
\usepackage{amsmath}
\usepackage{amssymb}

\makeatletter
%%%%%%%%%%%%%%%%%%%%%%%%%%%%%% User specified LaTeX commands.
\usepackage{tikz}
\usetikzlibrary{matrix,arrows,decorations.pathmorphing}
\usetikzlibrary{shapes.geometric}
\usepackage{tikz-cd}
\usepackage{amsthm}
\theoremstyle{plain}
\newtheorem{theorem}{Theorem}[section]
\newtheorem{lemma}[theorem]{Lemma}
\newtheorem{prop}{Proposition}[section]
\newtheorem*{cor}{Corollary}
\theoremstyle{definition}
\newtheorem{defn}{Definition}[section]
\newtheorem{ex}{Exercise} 
\newtheorem{example}{Example}[section]
\theoremstyle{remark}
\newtheorem*{rem}{Remark}
\newtheorem*{note}{Note}
\newtheorem{case}{Case}
\usepackage{graphicx}
\usepackage{amssymb}
\usepackage{tikz-cd}
\usetikzlibrary{calc,arrows,decorations.pathreplacing}
\tikzset{mydot/.style={circle,fill,inner sep=1.5pt},
commutative diagrams/.cd,
  arrow style=tikz,
  diagrams={>=latex},
}

\usepackage{babel}
\usepackage{hyperref}
\hypersetup{
    colorlinks,
    citecolor=black,
    filecolor=black,
    linkcolor=black,
    urlcolor=black
}
\usepackage{pgfplots}
\usetikzlibrary{decorations.markings}
\pgfplotsset{compat=1.9}

\makeatother

\usepackage{babel}
\begin{document}
\begin{theorem}\label{theorem} Suppose $I$ is a homogeneous ideal
generated by the set of polynomials $\{f_{1},f_{2},\dots,f_{s}\}$
such that $d(f_{i})=0$ for all $1\leq i\leq s$, and let $G$ be
the reduced Gr�bner basis with respect to the fixed monomial ordering. 
\begin{enumerate}
\item Then $d(g)=0$ for all $g\in G$. In particular, $\mathbf{A}_{\bullet}(R_{I,G})$
has a differential graded $k$-algebra structure with respect to the
multiplication maps in $R_{I,G}$. 
\item Moreover, let $G'$ be the reduced Gr�bner basis of $I$ with respect
to a different monomial ordering. Then $\mathbf{A}_{\bullet}(R_{I,G'})$
has a differential graded $k$-algebra structure with respect to the
multiplication maps in $R_{I,G'}$, and $\mathbf{A}_{\bullet}(R_{I,G'})$
is isomorphic to $\mathbf{A}_{\bullet}(R_{I,G})$ as differential
graded $k$-algebras. 
\end{enumerate}
\end{theorem}

\begin{proof} \hfill
\begin{enumerate}
\item Let $g\in G$. Since $g\in I$, we can write $g=q_{1}f_{1}+q_{2}f_{2}+\cdots+q_{s}f_{s}$
for some $q_{1},\dots,q_{s}\in R$. By the Leibnitz rule and our assumption,
we have
\[
d(g)=d(q_{1}f_{1}+q_{2}f_{2}+\cdots+q_{s}f_{s})=d(q_{1})f_{1}+d(q_{2})f_{2}+\cdots+d(q_{s})f_{s}.
\]
In particular, this implies $d(g)\in I$. Recall that $d(g)\in I$
if and only if $d(g)^{G}=0$. Combining this with Lemma 3.8, we find
that $d(g)=d(g)^{G}=0$. 
\item Applying the same argument as in (1) shows that $\mathbf{A}_{\bullet}(R_{I,G'})$
has a differential graded $k$-algebra structure with respect to the
multiplication maps in $R_{I,G'}$. The only nontrivial part to prove
is that $\mathbf{A}_{\bullet}(R_{I,G'})$ is isomorphic to $\mathbf{A}_{\bullet}(R_{I,G})$
as differential graded $k$-algebras. Recall that $R_{I,G}$ is isomorphic
as a graded $k$-algebra to $R/I$. The isomorphism is given by mapping
$f\in R_{I,G}$ to $\overline{f}\in R/I$. Similarly, $R/I$ is isomorphic
as a graded $k$-algebra to $R_{I,G'}$. The isomorphism is given
by mapping $\overline{f}\in R/I$ to $f^{G'}\in R_{I,G'}$. Combining
these two isomorphisms together gives us an isomorphism from $R_{I,G}$
to $R_{I,G'}$, given by mapping $f$ to $f^{G'}$. Thus, $R_{I,G}$
is isomorphic as a graded $k$-algebra to $R_{I,G'}$. In order for
~$\mathbf{A}_{\bullet}(R_{I,G})$ to be isomorphic to $\mathbf{A}_{\bullet}(R_{I,G'})$
as \emph{differential }graded $k$-algebras, we need this isomorphism
to commute with the differential $d$. That is, we need 
\[
d(f)^{G'}=d(f^{G'})\text{ for all }f\in R_{I,G}.
\]
This was already proven in Theorem 3.4. 
\end{enumerate}
\end{proof}
\end{document}
