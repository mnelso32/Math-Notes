%% LyX 2.2.3 created this file.  For more info, see http://www.lyx.org/.
%% Do not edit unless you really know what you are doing.
\documentclass[12pt,english]{article}
\usepackage[osf]{mathpazo}
\renewcommand{\sfdefault}{lmss}
\renewcommand{\ttdefault}{lmtt}
\usepackage[T1]{fontenc}
\usepackage[latin9]{inputenc}
\usepackage[paperwidth=30cm,paperheight=35cm]{geometry}
\geometry{verbose,tmargin=3cm,bmargin=3cm}
\setlength{\parindent}{0bp}
\usepackage{amsmath}
\usepackage{amssymb}

\makeatletter
%%%%%%%%%%%%%%%%%%%%%%%%%%%%%% User specified LaTeX commands.
\usepackage{tikz}
\usetikzlibrary{matrix,arrows,decorations.pathmorphing}
\usetikzlibrary{shapes.geometric}
\usepackage{tikz-cd}
\usepackage{amsthm}
\theoremstyle{plain}
\newtheorem{theorem}{Theorem}[section]
\newtheorem{lemma}[theorem]{Lemma}
\newtheorem{prop}{Proposition}[section]
\newtheorem*{cor}{Corollary}
\theoremstyle{definition}
\newtheorem{defn}{Definition}[section]
\newtheorem{ex}{Exercise} 
\newtheorem{example}{Example}[section]
\theoremstyle{remark}
\newtheorem*{rem}{Remark}
\newtheorem*{note}{Note}
\newtheorem{case}{Case}
\usepackage{graphicx}
\usepackage{amssymb}
\usepackage{tikz-cd}
\usetikzlibrary{calc,arrows,decorations.pathreplacing}
\tikzset{mydot/.style={circle,fill,inner sep=1.5pt},
commutative diagrams/.cd,
  arrow style=tikz,
  diagrams={>=latex},
}

\usepackage{babel}
\usepackage{hyperref}
\hypersetup{
    colorlinks,
    citecolor=black,
    filecolor=black,
    linkcolor=black,
    urlcolor=black
}
\usepackage{pgfplots}
\usetikzlibrary{decorations.markings}
\pgfplotsset{compat=1.9}

\makeatother

\usepackage{babel}
\begin{document}

\subsection*{Decomposing $H_{i}(S_{I})$ }

~~~Let $I$ be a homogeneous ideal and let $g$ be a homogeneous
polynomial of degree $j$. Let $G$ be the reduced Gr�bner basis for
$I$ and $G'$ be the reduced Gr�bner basis for $\langle I,g\rangle$
with respect to our fixed monomial ordering. In Commutative Algebra,
we learn about the following short exact sequence of graded $S$-modules\begin{center}\begin{tikzcd}[row sep=5] 0 \arrow[r] & (S/(I:g))(-j) \arrow[r, "g"] & S/I \arrow[r] & S/\langle I \text{,} g \rangle \arrow[r] & 0 

\\

& \overline{f} \arrow[r,mapsto,shorten >=0.5cm,shorten <=0.5cm] & \overline{fg}
\end{tikzcd}\end{center}

We want to use this short exact sequence to our advantage. First,
using the isomorphisms $S_{I:g}\cong S/(I:g)$, $S_{I}\cong S/I$,
and $S_{\langle I,g\rangle}\cong S/\langle I,g\rangle$, we get, for
each $i$, a short exact sequence of $K$-vector spaces

\begin{center}\begin{tikzcd}[row sep=5] 0 \arrow[r] & (S_{I:g} )_{j-i} \arrow[r, "\cdot g"] & (S_I )_i \arrow[r, "-^{G'} "] & ( S _{\langle I \text{,} g \rangle })_i \arrow[r] & 0 

\\

& f \arrow[r,mapsto,shorten >=0.5cm,shorten <=0.5cm] & (fg)^G

\\

&& f \arrow[r,mapsto,shorten >=0.5cm,shorten <=0.5cm] & f^{G' }
\end{tikzcd}\end{center}



or in other words, a short exact sequence of graded $K$-vector spaces 

\begin{center}\begin{tikzcd}[row sep=5] 0 \arrow[r] & (S_{I:g})(-j)  \arrow[r, "\cdot g"] & S_I  \arrow[r,"-^{G'} "] & S _{\langle I \text{,} g \rangle } \arrow[r] & 0

\end{tikzcd}\end{center}

We want to know under what conditions this becomes a short exact sequence
of chain complexes over $K$. That is, under what conditions when
does the following diagram commute?

\begin{center}\begin{tikzcd} & \vdots \arrow[d] & \vdots \arrow[d] & \vdots \arrow[d]

\\

0 \arrow[r] & (S_{I:g} )_{j-i} \arrow[d,"d",swap] \arrow[r, "\cdot g"] & (S_I )_i \arrow[d,"d",swap] \arrow[r, "-^{G'} "] & ( S _{\langle I \text{,} g \rangle })_i \arrow[r] \arrow[d,"d"] & 0 

\\

0 \arrow[r] & (S_{I:g} )_{j-i-1} \arrow[d] \arrow[r, "\cdot g"] & (S_I )_{i-1} \arrow[d] \arrow[r, "-^{G'} "] & ( S _{\langle I \text{,} g \rangle })_{i-1} \arrow[d] \arrow[r] & 0 

\\

& \vdots  & \vdots  & \vdots

\end{tikzcd}\end{center}



After some thought, we find that the conditions which need to be satisfied
are the following:
\begin{equation}
(gd(m))^{G}=d((gm)^{G})\text{ for all monomials }m\text{ which are not in }\text{LT}(I:g)\label{eq:condtion1}
\end{equation}
\begin{equation}
d(m)^{G'}=d(m^{G'})\text{ for all monomials }m\text{ which are not in }\text{LT}(I)\label{eq:condition2}
\end{equation}

For the moment, let's assume that these conditions are satisfied so
that we have a short exact sequence of chain complexes. Then by the
usual argument, the short exact sequence of chain complexes gives
rise to a long exact sequence in homology:

\begin{center}\begin{tikzcd}[row sep=40]  && \cdots \arrow[r] \arrow[d, phantom, ""{coordinate, name=Z'}] & H_{i+1} (S _{\langle I \text{,} g \rangle }) \arrow[dll, " \lambda ", swap, rounded corners, to path={ -- ([xshift=2ex]\tikztostart.east) |- (Z') [near end]\tikztonodes -| ([xshift=-2ex]\tikztotarget.west) -- (\tikztotarget)}] 



\\  & H_{i-j} (S_{I:g} ) \arrow[r, "\cdot g"] & H_{i} (S_I ) \arrow[r, "-^{G'} "] \arrow[d, phantom, ""{coordinate, name=Z}] & H_{i} (S _{\langle I \text{,} g \rangle }) \arrow[dll, " \lambda ", swap, rounded corners, to path={ -- ([xshift=2ex]\tikztostart.east) |- (Z) [near end]\tikztonodes -| ([xshift=-2ex]\tikztotarget.west) -- (\tikztotarget)}] 

\\ & H_{i-j-1} (S_{I:g} ) \arrow[r, "\cdot g "] & H_{i-1} (S_I ) \arrow[r, "-^{G'} "] & \cdots 

\end{tikzcd}\end{center}

It's easy to see that the connecting maps $\lambda$ all induce the
zero map. So in fact, we get for each $i$, the short exact sequence
of $K$-vector spaces:

\begin{center}\begin{tikzcd}[row sep=40] 0 \arrow[r]  & H_{i-j} (S_{I:g} ) \arrow[r, "\cdot g"] & H_{i} (S_I ) \arrow[r,"-^{G'} "] & H_{i} (S _{\langle I \text{,} g \rangle }) \arrow[r] & 0,  

\end{tikzcd}\end{center}

and since the inclusion map $S_{\langle I,g\rangle}\hookrightarrow S_{I}$
splits the map $-^{G'}$, we obtain the following isomorphism
\begin{equation}
H_{i-j}(S_{I:g})\oplus H_{i}(S_{\langle I,g\rangle})\cong H_{i}(S_{I})\label{eq:homologydirectsum}
\end{equation}

where we map the representative $(f_{1},f_{2})$ in $H_{i-j}(S_{I:g})\oplus H_{i}(S_{\langle I,g\rangle})$
to the representative $gf_{1}+f_{2}$ in $H_{i}(S_{I})$. We summarize
this discussion in the form of a theorem. 

\begin{theorem}\label{theoremdecomposition} Let $I$ be a homogeneous
ideal and let $g$ be a homogeneous polynomial of degree $j$. Let
$G$ be the reduced Gr�bner basis for $I$ and $G'$ be the reduced
Gr�bner basis for $\langle I,g\rangle$ with respect to our fixed
monomial ordering. Suppose that the following conditions are satisfied:
\begin{enumerate}
\item $(gd(m))^{G}=d((gm)^{G})\text{ for all monomials }m\text{ which are not in }\text{LT}(I:g)$.
\item $d(m)^{G'}=d(m^{G'})\text{ for all monomials }m\text{ which are not in }\text{LT}(I)$. 
\end{enumerate}
Then we have an isomorphism 
\[
H_{i-j}(S_{I:g})\oplus H_{i}(S_{\langle I,g\rangle})\cong H_{i}(S_{I})
\]
given by mapping the representative $(f_{1},f_{2})$ in $H_{i-j}(S_{I:g})\oplus H_{i}(S_{\langle I,g\rangle})$
to the representative $gf_{1}+f_{2}$ in $H_{i}(S_{I})$. \end{theorem}

\subsubsection*{Decomposition Example}

~~~We will now discuss a special case of when the conditions in
Theorem~(\ref{theoremdecomposition}) are satisfied. Consider the
case where $I$ is a monomial ideal and $g$ is a monomial of degree
$j$ which is not in $I$. Then condition (\ref{eq:condtion1}) is
satisfied since if $m$ is not in $I:g$, then $gm$ is not in $I$,
and so $(gm)^{G}=gm$ which implies $(gd(m))^{G}=gd(m)$. 

~~~For condition (\ref{eq:condition2}) first assume that $m$
is not in $\langle I,g\rangle$. Then then $m^{G'}=m$, which implies
$d(m)^{G'}=d(m)=d(m^{G'})$. Thus condition (\ref{eq:condition2})
is satisfied in this case. Now assume that $m=g$. Then $m^{G'}=0$,
which implies $d(m^{G'})=0$. Thus, we must have $d(g)=0$ in order
for condition (\ref{eq:condition2}) to be satisfied in this case.
So assume $d(g)=0$ and consider the final case where $m=m_{1}g$.
Since $d(g)=0$, we obtain $d(m)^{G'}=(d(m_{1})g)^{G'}=0$, and thus
(\ref{eq:condition2}) is satisfied in this case as well. 

~~~In the next example, we show how we can apply Theorem~(\ref{theoremdecomposition})
recursively. In what follows, we frequently use the notation $I,g$
to mean $\langle I,g\rangle$ and $I:g$ to mean $I:\langle g\rangle$.
For example, $I,g_{1}:g_{2}=\langle I,g_{1}\rangle:\langle g_{2}\rangle$,
and $I:g_{1},g_{2}=\langle(I:g_{1}),\langle g_{2}\rangle\rangle$,
and so on. We also note that $I:g_{1}:g_{2}=I:g_{1}g_{2}$. 

\begin{example}\label{examplerecursivehomology} Consider $S=K[x,y,z]$
and $I=\langle x^{3}y,yz^{3}\rangle$. Then $d(x^{2})=d(z^{2})=0$,
and so 
\begin{align*}
H_{i}(S_{I}) & =x^{2}H_{i-2}(S_{I:x^{2}})\oplus H_{i}(S_{I,x^{2}})\\
 & =x^{2}(z^{2}H_{i-4}(S_{I:x^{2}z^{2}})\oplus H_{i-2}(S_{I:x^{2},z^{2}}))\oplus z^{2}H_{i-2}(S_{I,x^{2}:z^{2}})\oplus H_{i}(S_{I,x^{2},z^{2}})\\
 & =x^{2}z^{2}H_{i-4}(S_{I:x^{2}z^{2}})\oplus x^{2}H_{i-2}(S_{I:x^{2},z^{2}})\oplus z^{2}H_{i-2}(S_{I,x^{2}:z^{2}})\oplus H_{i}(S_{I,x^{2},z^{2}})
\end{align*}

We calculate
\begin{align*}
I:x^{2}z^{2} & =\langle xy,yz\rangle\\
I,x^{2}:z^{2} & =\langle x^{2},yz\rangle\\
I:x^{2},z^{2} & =\langle xy,z^{2}\rangle\\
I,x^{2},z^{2} & =\langle x^{2},z^{2}\rangle
\end{align*}

The only part which has nontrivial homology is $S_{I:x^{2}z^{2}}$.
Thus, $H_{5}(S_{I})=[d(x^{3}yz^{2})]K$ and $H_{i}(S_{I})=0\text{ for }$all
$i\neq5$. \end{example}

\subsubsection*{More Differential Graded $K$-Algebras}

~~~To simplify notation, we often write ``$S_{I}$ is a differential
graded $K$-algebra'' when we really mean ``the chain complex $\mathbf{A}_{\bullet}(S_{I})$
can be given the structure of a differential graded $K$-algebra with
respect to multiplication maps in $S_{I}$'' . Let $I$ be a homogeneous
ideal such that $S_{I}$ is a differential graded $K$-algebra and
let $g$ be a homogeneous polynomial. If $d(g)=0$, then Theorem~(\ref{theoremdifferentialgradedalgebraI})
implies that $S_{\langle I,g\rangle}$ is also a differential graded
$K$-algebra. In this case, both condition (\ref{eq:condition1})
and condition (\ref{eq:condition2}) are satisfied. Indeed, let $G$
and $G'$ be the reduced Gr�bner bases of $I$ and $\langle I,g\rangle$
with respect to our fixed monomial ordering. Then since both $S_{I}$
and $S_{\langle I,g\rangle}$ are differential graded $K$-algebras,
we have $d(f)^{G}=d(f^{G})$ and $d(f)^{G'}=d(f^{G'})$ for any $f\in S$.
Since the homology of $S_{I}$ and $S_{\langle I,g\rangle}$ are trivial,
Theorem~(\ref{theoremdecomposition}) tells us that the homology of
$S_{I:g}$ is trivial too. Given that the homology of $S_{I:g}$ is
trivial, it's natural to wonder if $S_{I:g}$ is in fact a differential
graded $K$-algebra. It turns out it is:

\begin{theorem}\label{theorem} Let $I$ be a homogeneous ideal such
that $S_{I}$ is a differential graded $K$-algebra and let $g$ be
a homogeneous polynomial such that $d(g)=0$. Then both $S_{\langle I,g\rangle}$
and $S_{I:g}$ are also differential graded $K$-algebras, and we
have a short exact sequence of differential graded $K$-algebras:

\begin{center}\begin{tikzcd}[row sep=5] 0 \arrow[r] & S/(I:g) \arrow[r, "g"] & S/I \arrow[r] & S/\langle I \text{,} g \rangle \arrow[r] & 0 

\\

& \overline{f} \arrow[r,mapsto,shorten >=0.5cm,shorten <=0.5cm] & \overline{fg}
\end{tikzcd}\end{center}

\end{theorem}

\begin{proof} We've explained why $S_{\langle I,g\rangle}$ is a
differential graded $K$-algebra. So the only thing we need to show
is why $S_{I:g}$ is also a differential graded $K$-algebra. Let
$G''$ be the reduced Gr�bner basis of $I:g$ and let $g''$ be an
element in $G''$. We need to show that $d(g'')=0$. Since $g''\in I:g$,
we know that $gg''\in I$. So applying the division algorithm to $gg''$,
we obtain 
\begin{equation}
gg''=q_{1}g_{1}+\cdots+q_{r}g_{r}\label{eq:ggprimeprime}
\end{equation}
where $G=\{g_{1},\dots,g_{r}\}$ is the reduced Gr�bner basis for
$I$. Applying $d$ to both sides of (\ref{eq:ggprimeprime}) and
using the fact that $d(g)=0$ and $d(g_{\lambda})=0$ for all $g_{\lambda}\in G$,
we obtain
\[
gd(g'')=d(q_{1})g_{1}+\cdots+d(q_{r})g_{r}.
\]
Therefore, $d(g'')\in I:g$. Therefore $d(g'')=d(g'')^{G''}=0$. \end{proof}

\begin{example}\label{example} Consider $S=K[x,y,z]$, $g=x^{2}y+x^{2}z$,
and $I=\langle f_{1},f_{2},f_{3}\rangle$ where
\begin{align*}
f_{1} & =xy+xz+yz\\
f_{2} & =x^{4}y+x^{5}\\
f_{3} & =y^{3}+y^{2}z
\end{align*}
Then $d(f_{1})=d(f_{2})=d(f_{3})=0$ implies that $S_{I}$ is a differential
graded $K$-algebra. The reduced Gr�bner basis for $I$ with respect
to graded lexicographical ordering is $G=\{g_{1},g_{2},g_{3},g_{4},g_{5},g_{6}\}$,
where 
\begin{align*}
g_{1} & =xy+xz+yz\\
g_{2} & =y^{3}+y^{2}z\\
g_{3} & =y^{2}z^{2}\\
g_{4} & =xz^{4}+yz^{4}\\
g_{5} & =x^{5}+x^{4}z+x^{3}z^{2}+x^{2}z^{3}\\
g_{6} & =x^{4}z^{2}
\end{align*}

Since $d(g)=0$, we know that $S_{\langle I,g\rangle}$ and $S_{I:g}$
are also differential graded $K$-algebras. The reduced Gr�bner basis
for $I:g$ with respect to graded lexicographical ordering is $G''=\{g_{1}'',g_{2}'',g_{3}''\}$,
where 
\begin{align*}
g_{1}'' & =y+z\\
g_{2}'' & =z^{2}\\
g_{3}'' & =x^{3}+x^{2}z
\end{align*}
and the reduced Gr�bner basis for $\langle I,g\rangle$ with respect
to graded lexicographical ordering is $G'=\{g_{1}',g_{2}',g_{3}',g_{4}',g_{5}'\}$,
where 
\begin{align*}
g_{1}' & =xy+xz+yz\\
g_{2}' & =y^{3}+y^{2}z\\
g_{3}' & =xz^{2}+yz^{2}\\
g_{4}' & =y^{2}z^{2}\\
g_{5}' & =x^{5}+x^{4}z+x^{3}z^{2}+x^{2}z^{3}
\end{align*}

\end{example}
\end{document}
