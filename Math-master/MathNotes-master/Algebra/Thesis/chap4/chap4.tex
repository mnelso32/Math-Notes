%% LyX 2.2.3 created this file.  For more info, see http://www.lyx.org/.
%% Do not edit unless you really know what you are doing.
\documentclass[12pt,english]{article}
\usepackage[osf]{mathpazo}
\renewcommand{\sfdefault}{lmss}
\renewcommand{\ttdefault}{lmtt}
\usepackage[T1]{fontenc}
\usepackage[latin9]{inputenc}
\usepackage[paperwidth=30cm,paperheight=35cm]{geometry}
\geometry{verbose,tmargin=2cm,bmargin=2cm}
\setlength{\parindent}{0bp}
\usepackage{amsmath}
\usepackage{amssymb}

\makeatletter
%%%%%%%%%%%%%%%%%%%%%%%%%%%%%% User specified LaTeX commands.
\usepackage{tikz}
\usetikzlibrary{matrix,arrows,decorations.pathmorphing}
\usetikzlibrary{shapes.geometric}
\usepackage{tikz-cd}
\usepackage{amsthm}
\usepackage{xparse,etoolbox}

\theoremstyle{plain}
\newtheorem{theorem}{Theorem}[section]
\newtheorem{lemma}[theorem]{Lemma}
\newtheorem{prop}{Proposition}[section]
\newtheorem*{cor}{Corollary}
\theoremstyle{definition}
\newtheorem{defn}{Definition}[section]
\newtheorem{ex}{Exercise} 
\newtheorem{example}{Example}[section]
\theoremstyle{remark}
\newtheorem*{rem}{Remark}
\newtheorem*{note}{Note}
\newtheorem{case}{Case}
\usepackage{graphicx}
\usepackage{amssymb}
\usepackage{tikz-cd}
\usetikzlibrary{calc,arrows,decorations.pathreplacing}
\tikzset{mydot/.style={circle,fill,inner sep=1.5pt},
commutative diagrams/.cd,
  arrow style=tikz,
  diagrams={>=latex},
}

\usepackage{babel}
\usepackage{hyperref}
\hypersetup{
    colorlinks,
    citecolor=blue,
    filecolor=blue,
    linkcolor=blue,
    urlcolor=blue
}
\usepackage{pgfplots}
\usetikzlibrary{decorations.markings}
\pgfplotsset{compat=1.9}


\newcommand{\blocktheorem}[1]{%
  \csletcs{old#1}{#1}% Store \begin
  \csletcs{endold#1}{end#1}% Store \end
  \RenewDocumentEnvironment{#1}{o}
    {\par\addvspace{1.5ex}
     \noindent\begin{minipage}{\textwidth}
     \IfNoValueTF{##1}
       {\csuse{old#1}}
       {\csuse{old#1}[##1]}}
    {\csuse{endold#1}
     \end{minipage}
     \par\addvspace{1.5ex}}
}

\raggedbottom

\blocktheorem{theorem}% Make theo into a block
\blocktheorem{defn}% Make defi into a block
\blocktheorem{lemma}% Make lem into a block
\blocktheorem{rem}% Make rem into a block
\blocktheorem{cor}% Make col into a block
\blocktheorem{prop}% Make prop into a block


\usepackage[bottom]{footmisc}

\makeatother

\usepackage{babel}
\begin{document}

\title{Homological Constructions over a Ring of Characteristic 2}

\maketitle
~~~Throughout this chapter, let $R$ be a ring of characteristic
$2$. 

\section{Constructing All Finitely-Generated Differential Graded $R$-Algebras}

\begin{theorem}\label{theoremallfinitelygenerateddgalgebra} Let $S_{w}$
denote the weighted polynomial ring $R[x_{1},\dots,x_{n}]$ with respect
to the weighted vector $w=(w_{1},\dots,w_{n})$. Define the map 
\[
d:=\sum_{\lambda=1}^{n}f_{\lambda}\partial_{x_{\lambda}},
\]
where $f_{\lambda}$ is a nonzero homogeneous polynomial in $S_{w}$
of weighted degree $w_{\lambda}-1$ for all $\lambda=1,\dots,n$.
Then 
\begin{enumerate}
\item $d$ is a graded endomorphism $d:S_{w}\to S_{w}$ of degree $-1$
which satisfies Leibniz law. 
\item Moreover, let $I\subset S_{w}$ be any $d$-stable homogeneous ideal
such that $d(f_{\lambda})\in I$ for all $\lambda=1,\dots,n$. Then
$d$ induces a map $\overline{d}:S_{w}/I\to S_{w}/I$, given by $\overline{d}(\overline{f})=\overline{d(f)}$
for all $\overline{f}\in S_{w}/I$, and $(S_{w}/I,\overline{d})$
is a differential graded $R$-algebra.
\end{enumerate}
\end{theorem}

\begin{proof} We first show that $d$ is a graded endomorphism $d:S_{w}\to S_{w}$
of degree $-1$ which satisfies Leibniz law:
\begin{itemize}
\item $R$-linearity: We have 
\begin{align*}
d(r_{1}g_{1}+r_{2}g_{2}) & =\sum_{\lambda=1}^{n}f_{\lambda}\partial_{x_{\lambda}}(r_{1}g_{1}+r_{2}g_{2})\\
 & =\sum_{\lambda=1}^{n}f_{\lambda}(r_{1}\partial_{x_{\lambda}}(g_{1})+r_{2}\partial_{x_{\lambda}}(g_{2}))\\
 & =r_{1}\sum_{\lambda=1}^{n}f_{\lambda}\partial_{x_{\lambda}}(g_{1})+r_{2}\sum_{\lambda=1}^{n}f_{\lambda}\partial_{x_{\lambda}}(g_{2}))\\
 & =r_{1}d(g_{1})+r_{2}d(g_{2}),
\end{align*}
for all $r_{1},r_{2}\in R$ and $g_{1},g_{2}\in S_{w}$. 
\item Leibniz law: We have 
\begin{align*}
d(g_{1}g_{2}) & =\sum_{\lambda=1}^{n}f_{\lambda}\partial_{x_{\lambda}}(g_{1}g_{2})\\
 & =\sum_{\lambda=1}^{n}f_{\lambda}(\partial_{x_{\lambda}}(g_{1})g_{2}+g_{1}\partial_{x_{\lambda}}(g_{2}))\\
 & =\left(\sum_{\lambda=1}^{n}f_{\lambda}\partial_{x_{\lambda}}(g_{1})\right)g_{2}+g_{1}\left(\sum_{\lambda=1}^{n}f_{\lambda}\partial_{x_{\lambda}}(g_{2}))\right)\\
 & =d(g_{1})g_{2}+g_{1}d(g_{2}),
\end{align*}
for all $g_{1},g_{2}\in S_{w}$. 
\item Graded of degree $-1$: By $R$-linearity, we only need to check this
on monomials. Let $x_{1}^{\alpha_{1}}\cdots x_{n}^{\alpha_{n}}$ be
a monomial of weighted degree $i$. A term in $d(x_{1}^{\alpha_{1}}\cdots x_{n}^{\alpha_{n}})$
has the form $\alpha_{\lambda}f_{\lambda}x_{1}^{\alpha_{1}}\cdots x_{\lambda}^{\alpha_{\lambda}-1}\cdots x_{n}^{\alpha_{n}}$
where $\alpha_{\lambda}\equiv1\text{ mod }3$, and 
\begin{align*}
\text{deg}_{w}\left(\alpha_{\lambda}f_{\lambda}x_{1}^{\alpha_{1}}\cdots x_{\lambda}^{\alpha_{\lambda}-1}\cdots x_{n}^{\alpha_{n}}\right) & =\text{deg}_{w}\left(f_{\lambda}x_{1}^{\alpha_{1}}\cdots x_{\lambda}^{\alpha_{\lambda}-1}\cdots x_{n}^{\alpha_{n}}\right)\\
 & =\text{deg}_{w}\left(f_{\lambda}\right)+\text{deg}_{w}\left(x_{1}^{\alpha_{1}}\cdots x_{\lambda}^{\alpha_{\lambda}-1}\cdots x_{n}^{\alpha_{n}}\right)\\
 & =w_{\lambda}-1+w_{1}\alpha_{1}+\cdots+w_{\lambda}(\alpha_{\lambda}-1)+\cdots+w_{n}\alpha_{n}\\
 & =-1+w_{1}\alpha_{1}+\cdots+w_{\lambda}\alpha_{\lambda}+\cdots+w_{n}\alpha_{n}\\
 & =-1+i.
\end{align*}
So every term in $x_{1}^{\alpha_{1}}\cdots x_{n}^{\alpha_{n}}$ has
weighted degree $-1+i$. This implies that $d$ is graded of degree
$-1$. 
\end{itemize}
~~~Now we will show that $(S_{w}/I,\overline{d})$ is a differential
graded $R$-algebra. Since $I$ is $d$-stable, the map $\overline{d}$
is well-defined. The map $\overline{d}$ inherits the properties of
being a graded endomorphism of degree $-1$ which satisfies Leibniz
law from $d$, thus we just need to show that $\overline{d}^{2}=0$,
or in other words, that $d^{2}(g)\in I$ for all $g\in S_{w}$. So
let $g\in S_{w}$. Then
\begin{align*}
d^{2}(g) & =d\left(\sum_{\lambda=1}^{n}f_{\lambda}\partial_{x_{\lambda}}(g)\right)\\
 & =\sum_{\lambda=1}^{n}d(f_{\lambda}\partial_{x_{\lambda}}(g))\\
 & =\sum_{\lambda=1}^{n}d(f_{\lambda})\partial_{x_{\lambda}}(g)+f_{\lambda}d(\partial_{x_{\lambda}}(g)))\\
 & =\sum_{\lambda=1}^{n}d(f_{\lambda})\partial_{x_{\lambda}}(g)\in I,
\end{align*}
where we used the fact that $\partial_{x_{\lambda}}^{2}=0$ and $\partial_{x_{\mu}}\partial_{x_{\lambda}}=\partial_{x_{\lambda}}\partial_{x_{\mu}}$
to conclude that
\begin{align*}
\sum_{\lambda=1}^{n}f_{\lambda}d(\partial_{x_{\lambda}}(g)) & =\sum_{\lambda=1}^{n}f_{\lambda}\sum_{\mu=1}^{n}f_{\mu}\partial_{x_{\mu}}(\partial_{x_{\lambda}}(g))\\
 & =0.
\end{align*}

\end{proof}

\begin{rem} \hfill
\begin{enumerate}
\item We often denote this differential graded $R$-algebra as $(S_{w}/I,f_{1},\dots f_{n})$
instead of $(S_{w}/I,\overline{d})$. 
\item When we write ``let $(S_{w}/I,f_{1},\dots f_{n})$ be a differential
graded $R$-algebra'', it is understood that the conditions in Theorem~(\ref{theoremallfinitelygenerateddgalgebra})
are satisfied. Note that $I$ is a \emph{proper }ideal of $S_{w}$. 
\end{enumerate}
\end{rem}

\begin{prop}\label{prop} Let $(S_{w}/I,f_{1},\dots,f_{n})$ be a
differential graded $R$-algebra and let $g$ be a homogeneous polynomial
in $S$ of degree $j$ such that $d(g)$ is in $I$. Then $(S_{w}/\langle I,g\rangle,f_{1},\dots,f_{n})$
and $(S/(I:g),f_{1},\dots,f_{n})$ are differential graded $R$-algebras.
\end{prop}

\begin{proof} First note that $d(f_{\lambda})\in I$ implies $d(f_{\lambda})\in\langle I,g\rangle$
and $d(f_{\lambda})\in I:g$ for all $\lambda=1,\dots,n$. So we just
need to show that $\langle I,g\rangle$ and $I:g$ are $d$-stable.
Since$d(g)$ is in $I$, Proposition~(\ref{propcriteriondstable})
implies that $\langle I,g\rangle$ is $d$-stable. Therefore $S/\langle I,g\rangle$
is a differential graded $R$-algebra. To prove that $I:g$ is $d$-stable,
let $f\in I:g$. Then since $fg\in I$ and $I$ is $d$-stable, it
follows that $d(fg)=d(f)g+fd(g)\in I$. Which implies $d(f)g\in I$,
since $d(g)\in I$. Therefore $d(f)\in I:g$, which implies that $I:g$
is $d$-stable. \end{proof}

\subsection{Classification of all Finitely-Generated Commutative Differential
Graded $R$-Algebras}

\begin{theorem}\label{theoremclassificationdgalgebra} Every finitely-generated
commutative differential graded $R$-algebra is isomorphic to one
described in Theorem~(\ref{theoremallfinitelygenerateddgalgebra}).
\end{theorem}

\begin{proof} Let $(A,d_{A})$ be a finitely generated differential
graded $R$-algebra with generators $a_{1},\dots,a_{n}$. Then for
each $\lambda=1,\dots n$, we have $a_{\lambda}\in A_{w_{\lambda}}$,
where $w_{\lambda}\in\mathbb{Z}_{\geq0}$. Let $S_{w}$ denote the
weighted polynomial ring $R[x_{1},\dots,x_{n}]$ with respect to the
weighted vector $w=(w_{1},\dots,w_{n})$, and let $\varphi:S_{w}\to A$
be the unique morphism of graded $R$-algebras such that $\varphi(x_{\lambda})=a_{\lambda}$
for all $\lambda=1,\dots,n$. Then $A$ is isomorphic to $S_{w}/\text{Ker}(\varphi)$
as graded $R$-algebras. Choose $f_{\lambda}\in S$ such that $\varphi(f_{\lambda})=d_{A}(a_{\lambda})$
and define the map $d:S_{w}\to S_{w}$ as 
\[
d:=\sum_{\lambda=1}^{n}f_{\lambda}\partial_{x_{\lambda}}.
\]
Then $d$ is a graded endomorphism of degree $-1$ which satisfies
Leibniz law, by Theorem~(\ref{theoremallfinitelygenerateddgalgebra}).
We claim that $\text{Ker}(\varphi)$ is $d$-stable and that $d(f_{\lambda})\in\text{Ker}(\varphi)$
for all $\lambda=1,\dots,n$. We do this in two steps:

\hfill

\textbf{Step 1: }We will show that $\varphi d=d_{A}\varphi$. It suffices
to show that for all monomials $m$, we have $\varphi(d(m))=d_{A}(\varphi(m))$.
We prove this by induction on $\text{deg}(m)$. For the base case
$\text{deg}(m)=1$, we have $m=x_{\lambda}$ for some $\lambda\in\{1,\dots,n\}$.
Then 
\begin{align*}
\varphi(d(x_{\lambda})) & =\varphi(f_{\lambda})\\
 & =d_{A}(a_{\lambda})\\
 & =d_{A}(\varphi(x_{\lambda})).
\end{align*}
Now suppose that $\varphi(d(m))=d_{A}(\varphi(m))$ for all monomials
$m$ in $S$ of degree less than $i$, where $i>1$. Let $x_{1}^{\alpha_{1}}\cdots x_{n}^{\alpha_{n}}$
be a monomial in $S$ whose degree is $i+1$. We may assume that $\alpha_{1},\alpha_{\lambda}\geq1$
for some $\lambda\in\{1,\dots,n\}$. Then using Leibniz law together
with induction, we obtain
\begin{align*}
\varphi(d(x_{1}^{\alpha_{1}}x_{2}^{\alpha_{2}}\cdots x_{n}^{\alpha_{n}})) & =\varphi(d(x_{1}^{\alpha_{1}})x_{2}^{\alpha_{2}}\cdots x_{n}^{\alpha_{n}}+x_{1}^{\alpha_{1}}d(x_{2}^{\alpha_{2}}\cdots x_{n}^{\alpha_{n}}))\\
 & =\varphi(d(x_{1}^{\alpha_{1}})\varphi(x_{2}^{\alpha_{2}}\cdots x_{n}^{\alpha_{n}})+\varphi(x_{1}^{\alpha_{1}})\varphi(d(x_{2}^{\alpha_{2}}\cdots x_{n}^{\alpha_{n}}))\\
 & =\varphi(d(x_{1}^{\alpha_{1}}))a_{2}^{\alpha_{2}}\cdots a_{n}^{\alpha_{n}}+a_{1}^{\alpha_{1}}\varphi(d(x_{2}^{\alpha_{2}}\cdots x_{n}^{\alpha_{n}}))\\
 & =d_{A}(a_{1}^{\alpha_{1}})a_{2}^{\alpha_{2}}\cdots a_{n}^{\alpha_{n}}+a_{1}^{\alpha_{1}}d_{A}(a_{2}^{\alpha_{2}}\cdots a_{n}^{\alpha_{n}})\\
 & =d_{A}(a_{1}^{\alpha_{1}}a_{2}^{\alpha_{2}}\cdots a_{n}^{\alpha_{n}})\\
 & =d_{A}(\varphi(x_{1}^{\alpha_{1}}x_{2}^{\alpha_{2}}\cdots x_{n}^{\alpha_{n}})).
\end{align*}
This establishes Step 1. 

\hfill

\textbf{Step 2: }We show that $\text{Ker}(\varphi)$ is $d$-stable
and that $d(f_{\lambda})\in\text{Ker}(\varphi)$ for all $\lambda=1,\dots,n$.
Let $g\in\text{Ker}(\varphi)$. Then by Step 1, we have 
\begin{align*}
\varphi(d(f)) & =d_{A}(\varphi(f))\\
 & =d_{A}(0)\\
 & =0.
\end{align*}
Thus $d(f)\in\text{Ker}(\varphi)$, which implies $\text{Ker}(\varphi)$
is $d$-stable. Step 1 also implies 
\begin{align*}
\varphi(d(f_{\lambda})) & =d_{A}(\varphi(f_{\lambda}))\\
 & =d_{A}(d_{A}(f_{\lambda}))\\
 & =0,
\end{align*}
for all $\lambda=1,\dots,n$. 

\hfill

~~~Now Theorem~(\ref{theoremallfinitelygenerateddgalgebra}) implies
that $(S_{w}/\text{Ker}(\varphi),\overline{d})$ is a differential
graded $R$-algebra. Moreover, Step 1 implies $\varphi:(S_{w}/\text{Ker}(\varphi),\overline{d})\to(A,d_{A})$
is an isomorphism of differential graded $R$-algebras. \end{proof}

\section{Constructing the Differential Graded $R$-algebra $(S/I,r_{1},\dots,r_{n})$}

~~~We now want to consider some special cases of Theorem~(\ref{theoremallfinitelygenerateddgalgebra}).
In particular, we want to consider the case where the weighted vector
is $w=(1,\dots,1)$. We will write $S$ to denote the polynomial ring
$R[x_{1},\dots,x_{n}]$ equipped with this grading. Let $r_{1},\dots,r_{n}$
be nonzero elements in $R$, and define $d:S\to S$ by 
\[
d:=\sum_{\lambda=1}^{n}r_{\lambda}\partial_{x_{\lambda}}.
\]
Since $d(r_{\lambda})=0$ for all $\lambda=1,\dots,n$, it follows
from Theorem~(\ref{theoremallfinitelygenerateddgalgebra}) that $(S,r_{1},\dots,r_{n})$
is a differential graded $R$-algebra. Moreover, if $I$ is a $d$-stable
ideal, then $(S/I,r_{1},\dots,r_{n})$ is a differential graded $R$-algebra.
The next proposition gives a necessary and sufficient condition for
a finitely generated ideal $I$ to be $d$-stable.

\begin{prop}\label{propcriteriondstable} Let $I$ be a homogeneous
ideal in $S$. Then $I$ is $d$-stable if and only if for some generating
set $F=\{f_{1},\dots,f_{r}\}$ of $I$, we have $d(f_{\lambda})\in I$
for all $\lambda=1,\dots,r$. \end{prop}

\begin{proof} One direction is trivial, so let's prove the other
direction. Let $F=\{f_{1},\dots,f_{r}\}$ be a generating set for
$I$ such that $d(f_{\lambda})\in I$ for all $\lambda=1,\dots,r$
and let $f\in I$. Since $\{f_{1},\dots,f_{r}\}$ generates $I$,
we can write $f=\sum_{\lambda=1}^{r}q_{\lambda}f_{\lambda}$ for some
$q_{1},\dots,q_{r}\in S$. Thus, by Leibniz law, we have
\begin{align*}
d(f) & =d\left(\sum_{\lambda=1}^{r}q_{\lambda}f_{\lambda}\right)\\
 & =\sum_{\lambda=1}^{r}d(q_{\lambda}f_{\lambda})\\
 & =\sum_{\lambda=1}^{r}(d(q_{\lambda})f_{\lambda}+q_{\lambda}d(f_{\lambda}))\in I.
\end{align*}
Thus, $I$ is $d$-stable. \end{proof} 

\subsection{Koszul Complex}

~~~Recall from Example~(\ref{examplekoszulcomplex}) that the Koszul
complex $\mathcal{K}(r_{1},\dots,r_{n})$ is a differential graded
$R$-algebra. Indeed, $\mathcal{K}(r_{1},\dots,r_{n})$ is isomorphic
to the differential graded $R$-algebra $(S/I,r_{1},\dots,r_{n})$,
where $I$ is generated by $\left\{ x_{1}^{2},\dots,x_{n}^{2}\right\} $.C
learly $I$ is $d$-stable since $d(x_{\lambda}^{2})=0$ for all $\lambda=1,\dots,n$. 

\begin{example}\label{example} Let $R=\mathbb{F}_{2}[x,y]/\langle xy\rangle$
and let $r_{1}=x$ and $r_{2}=y$. Then $S=R[u,v]$ has a differential
graded $R$-algebra structure with the differential $d$ given by
\[
d:=x\partial_{u}+y\partial_{v}.
\]
Using graded lexicographical ordering on the monomials, we can explicitly
write $S$ as a chain complex over $R$ using matrices as the linear
maps:

\begin{center}\begin{tikzcd}[ampersand replacement=\&] \cdots  \arrow[r] \& R^4 \arrow{rrrr}{\begin{pmatrix} x & y & 0 & 0 \\ 0 & 0 & 0 & 0 \\ 0 & 0 & x & y \end{pmatrix}} \& \& \& \& R^3 \arrow{rrr}{\begin{pmatrix} 0 & y & 0 \\ 0 & x & 0 \end{pmatrix}}  \& \& \& R^2 \arrow{rr}{\begin{pmatrix} x & y \end{pmatrix}}  \& \& R \arrow[r] \& 0  \end{tikzcd}\end{center}

~~~Now let $I$ be the homogeneous ideal in $S$ generated by $\{x^{2},y^{2}\}$.
Then $(S/I,r_{1},r_{2})$ is isomorphic to the Koszul complex $\mathcal{K}(r_{1},r_{2})$.
Using graded lexicographical ordering on the monomials, we can explicitly
write $S/I$ as a chain complex over $R$ using matrices as the linear
maps:

\begin{center}\begin{tikzcd}[ampersand replacement=\&] 0 \arrow[r] \& R \arrow{rr}{\begin{pmatrix} y \\ x \end{pmatrix}}   \& \& R^2 \arrow{rr}{\begin{pmatrix} x & y \end{pmatrix}}  \& \& R \arrow[r] \& 0  \end{tikzcd}\end{center}

\end{example}

\subsection{Blowup algebras}

\begin{prop}\label{propblowupalgaisdgalg} Let $Q$ be a finitely
generated ideal in $R$ with generating set $\{a_{1},\dots,a_{n}\}$.
Then the blowup algebra $B_{Q}(R)$ can be given the structure of
differential graded $R$-algebra. \end{prop}

\begin{proof} Let $\varphi:S\to B_{Q}(R)$ be the unique graded $R$-algebra
homomorphism such that $\varphi(x_{\lambda})=ta_{\lambda}$ for all
$\lambda=1,\dots,n$ and let $d:=\sum_{\lambda=1}^{n}a_{\lambda}\partial_{\lambda}$.
We claim that $\text{Ker}(\varphi)$ is $d$-stable. Indeed, let $f\in\text{Ker}(\varphi)$.
Since $\text{Ker}(\varphi)$ is homogeneous, we may assume that $f$
is homogeneous. Write $f$ and $d(f)$ in terms of the monomial basis:
\[
f=\sum_{\lambda=1}^{r}b_{\lambda}x_{1}^{\alpha_{1\lambda}}\cdots x_{n}^{\alpha_{n\lambda}}\qquad\text{and}\qquad d(f)=\sum_{\substack{1\leq\mu\leq n\\
1\leq\lambda\leq r
}
}\alpha_{\mu\lambda}a_{\mu}b_{\lambda}x_{1}^{\alpha_{1\lambda}}\cdots x_{\mu}^{\alpha_{\mu\lambda}-1}\cdots x_{n}^{\alpha_{n\lambda}}.
\]
where $b_{\lambda}\in R$ and $\alpha_{\mu\lambda}\in\mathbb{Z}_{\geq0}$
for all $\lambda=1,\dots,r$ and $\mu=1,\dots n$. Observe that 
\begin{align*}
0 & =\varphi(f)\\
 & =\varphi\left(\sum_{\lambda=1}^{r}b_{\lambda}x_{1}^{\alpha_{1\lambda}}\cdots x_{n}^{\alpha_{n\lambda}}\right)\\
 & =\sum_{\lambda=1}^{r}b_{\lambda}\varphi(x_{1})^{\alpha_{1\lambda}}\cdots\varphi(x_{n})^{\alpha_{n\lambda}}\\
 & =t^{i}\left(\sum_{\lambda=1}^{r}b_{\lambda}a_{1}^{\alpha_{1\lambda}}\cdots a_{n}{}^{\alpha_{n\lambda}}\right)
\end{align*}
implies that $\sum_{\lambda=1}^{r}b_{\lambda}a_{1}^{\alpha_{1\lambda}}\cdots a_{n}{}^{\alpha_{n\lambda}}=0$.
Therefore
\begin{align*}
\varphi(d(f)) & =\varphi\left(\sum_{\substack{1\leq\mu\leq n\\
1\leq\lambda\leq r
}
}\alpha_{\mu\lambda}a_{\mu}b_{\lambda}x_{1}^{\alpha_{1\lambda}}\cdots x_{\mu}^{\alpha_{\mu\lambda}-1}\cdots x_{n}^{\alpha_{n\lambda}}\right)\\
 & =\sum_{\substack{1\leq\mu\leq n\\
1\leq\lambda\leq r
}
}\alpha_{\mu\lambda}a_{\mu}b_{\lambda}\varphi(x_{1})^{\alpha_{1\lambda}}\cdots\varphi(x_{\mu})^{\alpha_{\mu\lambda}-1}\cdots\varphi(x_{n})^{\alpha_{n\lambda}}\\
 & =t^{i-1}\left(\sum_{\substack{1\leq\mu\leq n\\
1\leq\lambda\leq r
}
}\alpha_{\mu\lambda}a_{\mu}b_{\lambda}a_{1}^{\alpha_{1\lambda}}\cdots a_{\mu}^{\alpha_{\mu\lambda}-1}\cdots a_{n}^{\alpha_{n\lambda}}\right)\\
 & =t^{i-1}\left(\left(\sum_{\mu=1}^{n}\alpha_{\mu\lambda}\right)\left(\sum_{\lambda=1}^{r}b_{\lambda}a_{1}^{\alpha_{1\lambda}}\cdots a_{n}{}^{\alpha_{n\lambda}}\right)\right)\\
 & =0.
\end{align*}
Therefore $(S/\text{Ker}(\varphi),a_{1},\dots,a_{n})$ is a differential
graded $R$-algebra where $S/\text{Ker}(\varphi)\cong B_{Q}(R)$.
\end{proof}

\begin{rem}\label{rem} It isn't too difficult to show that this differential
graded $R$-algebra is $(B_{Q}(R),\partial_{t})$, where $\partial_{t}$
is defined in the obvious way. \end{rem}

\begin{example}\label{exampleblowupalgebra} Let $R=\mathbb{F}_{2}[x,y]/\langle y^{2}+x^{3}+x^{2}\rangle$,
$\mathfrak{m}$ be the maximal ideal in $R$ generated by $\{\overline{x},\overline{y}\}$,
$S$ denote the polynomial ring $R[u,v]$, and $d=\overline{x}\partial_{u}+\overline{y}\partial_{v}$.
There is a surjective $R$-algebra homomorphism from $S$ to the blowup
algebra at $\mathfrak{m}$ given by
\[
\varphi:S:=\mathbb{F}_{2}[x,y,u,v]/\langle y^{2}+x^{3}+x^{2}\rangle\to B_{\mathfrak{m}}(R),
\]
where $\varphi$ is induced by $\varphi(u)=t\overline{x}$ and $v\mapsto t\overline{y}$.
Using Singular, we find that the kernel of $\varphi$ is an ideal
which is homogeneous in the variables $u,v,$ and is generated by
the set $\{f_{1},f_{2},f_{3}\}$, where 
\begin{align*}
f_{1} & =\overline{x}v+\overline{y}u\\
f_{2} & =\overline{x}u^{2}+u^{2}+v^{2}\\
f_{3} & =\overline{x}^{2}u+\overline{x}u+\overline{y}v
\end{align*}
Note that $d(f_{1})=d(f_{2})=d(f_{3})\in\text{Ker}(\varphi)$. It
follows from Proposition~(\ref{propcriteriondstable}) that $\text{Ker}(\varphi)$
is $d$-stable, which we already knew from Proposition~(\ref{propblowupalgaisdgalg}).
\end{example}

\subsection{Homology Calculations}

\begin{prop}\label{propSmodIr1rnisfree} Let $(S/I,r_{1},\dots,r_{n})$
be a differential graded $R$-algebra. Suppose that there are \textbf{$t_{1},\dots,t_{n}\in R$
}such 
\begin{equation}
\sum_{\lambda=1}^{n}t_{\lambda}r_{\lambda}=1.\label{eq:unitideal}
\end{equation}
Then $H(S/I,r_{1},\dots,r_{n})=0$. \end{prop}

\begin{proof} First note that $\sum_{\lambda=1}^{n}t_{\lambda}x_{\lambda}\notin I$,
otherwise $d\left(\sum_{\lambda=1}^{n}t_{\lambda}x_{\lambda}\right)=1\notin I$
would imply that $I$ is not $d$-stable. Let $f$ be a homogeneous
polynomial of degree $i$ such $d(f)\in I$; so $f$ represents a
cycle of $(S/I,\overline{d})$. Then
\begin{align*}
d\left(\left(\sum_{\lambda=1}^{n}t_{\lambda}x_{\lambda}\right)f\right) & =d\left(\sum_{\lambda=1}^{n}t_{\lambda}x_{\lambda}\right)f+\left(\sum_{\lambda=1}^{n}t_{\lambda}x_{\lambda}\right)d(f)\\
 & =\left(\sum_{\lambda=1}^{n}t_{\lambda}r_{\lambda}\right)f+\left(\sum_{\lambda=1}^{n}t_{\lambda}x_{\lambda}\right)d(f)\\
 & =f+\left(\sum_{\lambda=1}^{n}t_{\lambda}x_{\lambda}\right)d(f)\\
 & \equiv f\text{ mod }I.
\end{align*}
thus $\text{Ker}(\overline{d})=\text{Im}(\overline{d})$, which proves
the claim. \end{proof}

\begin{rem}\label{remSIisfreeforspecialG} \hfill
\begin{enumerate}
\item By setting $I=0$, we also find that $H(S)=0$. 
\item The condition (\ref{eq:unitideal}) is equivalent to saying that $\{r_{1},\dots,r_{n}\}$
generates the unit ideal. 
\end{enumerate}
\end{rem}

\subsubsection{Long Exact Sequence}

~~~It is straightforward to check that 

\begin{equation}\label{sesdga}\begin{tikzcd}[row sep=5] 0 \arrow[r] & (S_w (-j) /(I:g) \text{,}\overline{d} ) \arrow[r, "\cdot g"] & (S/I \text{,} \overline{d}) \arrow[r] & (S/\langle I \text{,} g \rangle \text{,} \overline{d}) \arrow[r] & 0 \\ & \overline{f} \arrow[r,mapsto,shorten >=0.5cm,shorten <=0.5cm] & \overline{fg} \end{tikzcd}\end{equation}

is short exact sequence of chain complexes. The short exact sequence
(\ref{sesdga}) gives rise to a long exact sequence in homology:

\begin{center}\begin{tikzcd}[row sep=40]  && \cdots \arrow[r] \arrow[d, phantom, ""{coordinate, name=Z'}] & H_{i+1} (S_w  / \langle I \text{,} g \rangle  ) \arrow[dll, " \lambda  ", swap, rounded corners, to path={ -- ([xshift=2ex]\tikztostart.east) |- (Z') [near end]\tikztonodes -| ([xshift=-2ex]\tikztotarget.west) -- (\tikztotarget)}] 



\\  & H_{i-j} (S_w /( I:g )) \arrow[r, "\cdot g"] & H_{i} (S_w / I) \arrow[r] \arrow[d, phantom, ""{coordinate, name=Z}] & H_{i} (S_w / \langle I \text{,} g \rangle  ) \arrow[dll, " \lambda ", swap, rounded corners, to path={ -- ([xshift=2ex]\tikztostart.east) |- (Z) [near end]\tikztonodes -| ([xshift=-2ex]\tikztotarget.west) -- (\tikztotarget)}] 

\\ & H_{i-j-1} (S_w /( I:g ) ) \arrow[r, "\cdot g "] & H_{i-1} (S_w / I ) \arrow[r] & \cdots 

\end{tikzcd}\end{center}

~~~Let us work out the details of the connecting map: Let $\overline{f}$
be a homogeneous element in $S_{w}/\langle I,g\rangle$ which represents
a class in $H_{i}(S_{w}/\langle I,g\rangle)$. In particular, $f\in S$
and $d(f)\in\langle I,g\rangle$. We lift $\overline{f}\in S_{w}/\langle I,g\rangle$
to $S_{w}/I$ and then apply $d$ to get $\overline{d(f)}\in S_{w}/I$.
Since $d(f)\in\langle I,g\rangle$, we can write $d(f)=p+gq$ where
$p\in I$. Thus, $\overline{d(f)}=\overline{gq}$, and this pulls
back to $\overline{q}$ in $S_{w}/(I:g)$. 

\section{Extra}

\subsection{Classifying $d$-Stable Ideals}

~~~Let $(R[x_{1},\dots,x_{n}]/I,r_{1},\dots,r_{n})$ be a differential
graded $R$-algebra. Suppose that there are \textbf{$t_{1},\dots,t_{m}\in R$
}such that $\langle r_{1},\dots,r_{n}\rangle=\langle t_{1},\dots,t_{m}\rangle$
and $(R[y_{1},\dots,y_{m}]/I,t_{1},\dots,t_{m})$ is also a differential
graded $R$-algebra. Then for all $1\leq\lambda\leq n$ and $1\leq\mu\leq n$,
there are $a_{\lambda\mu}$ and $b_{\lambda\mu}$ in $R$ such that
\[
r_{\lambda}=\sum_{\mu=1}^{m}a_{\lambda\mu}t_{\mu}\text{ and }t_{\mu}=\sum_{\lambda=1}^{n}b_{\lambda\mu}r_{\lambda}.
\]
~~~Let $\varphi:R[x_{1},\dots,x_{n}]\to R[y_{1},\dots,y_{m}]$
be the unique graded $R$-algebra homomorphism such that $\varphi(x_{\lambda})=\sum_{\mu=1}^{m}a_{\lambda\mu}y_{\mu}$
for all $\lambda=1,\dots,n$. Then $\varphi$ induces a graded $R$-algebra
homomorphism $\overline{\varphi}:R[x_{1},\dots,x_{n}]/I\to R[y_{1},\dots,y_{m}]/\langle\varphi(I)\rangle$
which in turn induces a homomorphism of differential graded $R$-algebras
$\overline{\varphi}:(R[x_{1},\dots,x_{n}]/I,r_{1},\dots,r_{n})\to(R[y_{1},\dots,y_{m}]/\langle\varphi(I)\rangle,t_{1},\dots,t_{m})$.
Indeed, let us denote the differentials as
\[
d_{r}:=\sum_{\lambda=1}^{n}r_{\lambda}\partial_{x_{\lambda}}\text{ and }d_{t}:=\sum_{\mu=1}^{m}t_{\mu}\partial_{y_{\mu}}.
\]

We first show that $\varphi d_{r}=d_{t}\varphi$. It is enough to
show that $\varphi d_{r}(x_{\lambda})=d_{t}\varphi(x_{\lambda})$
for all $\lambda=1,\dots,n$. We have 
\begin{align*}
d_{t}\varphi(x_{\lambda}) & =d_{t}\left(\sum_{\mu=1}^{m}a_{\lambda\mu}y_{\mu}\right)\\
 & =\sum_{\mu=1}^{m}a_{\lambda\mu}t_{\mu}\\
 & =r_{\lambda}\\
 & =d_{r}(x_{\lambda})\\
 & =\varphi(d_{r}(x_{\lambda})).
\end{align*}
Now we show that $(R[y_{1},\dots,y_{m}]/\langle\varphi(I)\rangle,t_{1},\dots,t_{m})$
is a differential graded $R$-algebra. We do this by showing that
$\langle\varphi(I)\rangle$ is $d_{t}$-stable. Let $\sum_{\kappa=1}^{r}g_{\kappa}\varphi(f_{\kappa})\in\varphi(I)$.
Then
\begin{align*}
d_{t}\left(\sum_{\kappa=1}^{r}g_{\kappa}\varphi(f_{\kappa})\right) & =\sum_{\kappa=1}^{r}d_{t}(g_{\kappa})\varphi(f_{\kappa})+\sum_{\kappa=1}^{r}g_{\kappa}d_{t}(\varphi(f_{\kappa}))\\
 & =\sum_{\kappa=1}^{r}d_{t}(g_{\kappa})\varphi(f_{\kappa})+\sum_{\kappa=1}^{r}g_{\kappa}\varphi(d_{r}(f_{\kappa}))\in\langle\varphi(I)\rangle.
\end{align*}

Similarly, let $\psi:R[y_{1},\dots,y_{m}]\to R[x_{1},\dots,x_{n}]$
be the unique graded $R$-algebra homomorphism such that $\psi(y_{\mu})=\sum_{\lambda=1}^{n}b_{\lambda\mu}x_{\lambda}$
for all $\mu=1,\dots,m$. Then $\varphi$ induces a graded $R$-algebra
homomorphism $\overline{\psi}:R[y_{1},\dots,y_{m}]/\langle\varphi(I)\rangle\to R[x_{1},\dots,x_{n}]/\langle\psi(\varphi(I))\rangle$
which in turn induces a homomorphism of differential graded $R$-algebras
$\overline{\psi}(R[y_{1},\dots,y_{m}]/\langle\varphi(I)\rangle,t_{1},\dots,t_{m})\to(R[x_{1},\dots,x_{n}]/\langle\psi(\varphi(I))\rangle,r_{1},\dots,r_{n})$. 

\subsubsection{Evalutation Map}

~~~Let $(S/I,r_{1},\dots,r_{n})$ be a differential graded $R$-algebra
such that $I$ is contained in $\langle x_{1},\dots,x_{n}\rangle$.
Let $Q=\langle r_{1},\dots,r_{n}\rangle$ and $\text{Ev}_{r}:S\to R$
be the unique $R$-algebra homomorphism such that $\text{Ev}_{r}(x_{\lambda})=r_{\lambda}$
for all $\lambda=1,\dots,n$. We are interested in the ideal $\text{Ev}_{r}(I)$
in $R$. Clearly we have $\text{Ev}_{r}(I)\subset Q$. Suppose $a\in Q\backslash\text{Ev}_{r}(I)$.
Then $a=\sum_{\lambda=1}^{n}a_{\lambda}r_{\lambda}$ for some $a_{\lambda}\in R$.
This implies $x:=\sum_{\lambda=1}^{n}a_{\lambda}x_{\lambda}\notin I$.
Now $J=I+\langle x,a\rangle$ is an ideal strictly larger than $I$
such that $J$ is $d$-stable $\text{Ev}_{r}(J)$ is strictly larger
than $\text{Ev}_{r}(I)$. This implies that we can find an ideal $I$
such that $\text{Ev}_{r}(I)=Q$.

\begin{prop}\label{prop} Let $(S/I,r_{1},\dots,r_{n})$ be a differential
graded $R$-algebra and let $Q=\langle r_{1},\dots,r_{n}\rangle$
be an ideal in $R$. Suppose that $\text{Ev}_{r}(I)=0$. Then there
exists a unique homomorphism $\varphi$ which makes the following
diagram commute

\begin{center}\begin{tikzcd}  & B_Q (R) \arrow[dr, "\text{Ev} _1 "] 

\\ 

S/I \arrow[ur,"\varphi " ,dashrightarrow] \arrow[rr, "\text{Ev} _r " , swap] && R

\end{tikzcd}\end{center}

\end{prop}

\subsubsection{Tensor product of differential graded $R$-algebras}

~~~Let $(R[x_{1},\dots,x_{n}]/I,d_{r})$ and $(R[y_{1},\dots,y_{m}]/J,d_{t})$
be two differential graded $R$-algebras, where 
\[
d_{r}:=\sum_{\lambda=1}^{n}r_{\lambda}\partial_{x_{\lambda}}\text{ and }d_{t}:=\sum_{\mu=1}^{m}t_{\mu}\partial_{y_{\mu}}.
\]
for $r_{\lambda},t_{\mu}\in R$ for all $\lambda=1,\dots,n$ and $\mu=1,\dots,m$.
Then their tensor product over $R$ is 
\[
(R[x_{1},\dots,x_{n}]/I,d_{r})\otimes_{R}(R[y_{1},\dots,y_{m}]/J,d_{t})\cong(R[x_{1},\dots,x_{n},y_{1},\dots,y_{m}]/(I+J),d_{r}+d_{t}).
\]

\begin{example}\label{example} The Koszul complex $\mathcal{K}(r_{1},\dots,r_{n})$
can be realized as a tensor product:
\[
\mathcal{K}(r_{1},\dots,r_{n})\cong\mathcal{K}(r_{1})\otimes\cdots\otimes\mathcal{K}(r_{n}).
\]

\end{example} 

~~~Let $M$ be an $R$-module, and let $(S/I,r_{1},\dots,r_{n})$
be a differential graded $R$-algebra. Recall that $(M\otimes_{R}S/I,d)$
is an $(S/I)$-module. 
\end{document}
