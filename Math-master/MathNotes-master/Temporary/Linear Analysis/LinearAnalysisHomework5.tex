%% LyX 2.3.2-2 created this file.  For more info, see http://www.lyx.org/.
%% Do not edit unless you really know what you are doing.
\documentclass[12pt,english]{article}
\usepackage[osf]{mathpazo}
\renewcommand{\sfdefault}{lmss}
\renewcommand{\ttdefault}{lmtt}
\usepackage[T1]{fontenc}
\usepackage[latin9]{inputenc}
\usepackage[paperwidth=30cm,paperheight=35cm]{geometry}
\geometry{verbose,tmargin=2cm,bmargin=2cm}
\setlength{\parindent}{0bp}
\usepackage{amsmath}
\usepackage{amssymb}

\makeatletter
\@ifundefined{date}{}{\date{}}
%%%%%%%%%%%%%%%%%%%%%%%%%%%%%% User specified LaTeX commands.
\usepackage{tikz}
\usetikzlibrary{matrix,arrows,decorations.pathmorphing}
\usetikzlibrary{shapes.geometric}
\usepackage{tikz-cd}
\usepackage{amsthm}
\usepackage{xparse,etoolbox}

\theoremstyle{plain}
\newtheorem{theorem}{Theorem}[section]
\newtheorem{lemma}[theorem]{Lemma}
\newtheorem{prop}{Proposition}[section]
\newtheorem*{cor}{Corollary}
\theoremstyle{definition}
\newtheorem{defn}{Definition}[section]
\newtheorem{ex}{Exercise} 
\newtheorem{example}{Example}[section]
\theoremstyle{remark}
\newtheorem*{rem}{Remark}
\newtheorem*{note}{Note}
\newtheorem{case}{Case}
\usepackage{graphicx}
\usepackage{amssymb}
\usepackage{tikz-cd}
\usetikzlibrary{calc,arrows,decorations.pathreplacing}
\tikzset{mydot/.style={circle,fill,inner sep=1.5pt},
commutative diagrams/.cd,
  arrow style=tikz,
  diagrams={>=latex},
}

\usepackage{babel}
\usepackage{hyperref}
\hypersetup{
    colorlinks,
    citecolor=blue,
    filecolor=blue,
    linkcolor=blue,
    urlcolor=blue
}
\usepackage{pgfplots}
\usetikzlibrary{decorations.markings}
\pgfplotsset{compat=1.9}


\newcommand{\blocktheorem}[1]{%
  \csletcs{old#1}{#1}% Store \begin
  \csletcs{endold#1}{end#1}% Store \end
  \RenewDocumentEnvironment{#1}{o}
    {\par\addvspace{1.5ex}
     \noindent\begin{minipage}{\textwidth}
     \IfNoValueTF{##1}
       {\csuse{old#1}}
       {\csuse{old#1}[##1]}}
    {\csuse{endold#1}
     \end{minipage}
     \par\addvspace{1.5ex}}
}

\raggedbottom

\blocktheorem{theorem}% Make theo into a block
\blocktheorem{defn}% Make defi into a block
\blocktheorem{lemma}% Make lem into a block
\blocktheorem{rem}% Make rem into a block
\blocktheorem{cor}% Make col into a block
\blocktheorem{prop}% Make prop into a block


\usepackage[bottom]{footmisc}

\makeatother

\usepackage{babel}
\begin{document}
\title{Linear Analysis Homework 5}
\author{Michael Nelson}

\maketitle
Throughout this homework, let $\mathcal{H}$ be a Hilbert space. 

\subsection*{Problem 1}

\begin{prop}\label{prop} Let $T\colon\mathcal{H}\to\mathcal{H}$
and $S\colon\mathcal{H}\to\mathcal{H}$ be two bounded operators.
Then
\begin{equation}
(\alpha T+\beta S)^{*}=\overline{\alpha}T^{*}+\overline{\beta}S^{*}\label{eq:adjointeq}
\end{equation}
for all $\alpha,\beta,\in\mathbb{C}$. \end{prop}

\begin{proof} Let $\alpha,\beta,\in\mathbb{C}$ and let $y\in\mathcal{H}$.
Then for all $x\in\mathcal{H}$, we have
\begin{align*}
\langle x,(\alpha T+\beta S)^{*}y\rangle & =\langle(\alpha T+\beta S)x,y\rangle\\
 & =\alpha\langle Tx,y\rangle+\beta\langle Sx,y\rangle\\
 & =\alpha\langle x,T^{*}y\rangle+\beta\langle x,S^{*}y\rangle\\
 & =\langle x,(\overline{\alpha}T^{*}+\overline{\beta}S^{*})y\rangle
\end{align*}
In particular, this implies $(\alpha T+\beta S)^{*}y=(\overline{\alpha}T^{*}+\overline{\beta}S^{*})y$
for all $y\in\mathcal{H}$ (by positive-definiteness of the inner-product)
which implies (\ref{eq:adjointeq}). \end{proof}

\subsection*{Problem 2}

\begin{prop}\label{propcomp} Let $T\colon\mathcal{H}\to\mathcal{H}$
and $S\colon\mathcal{H}\to\mathcal{H}$ be two bounded operators.
Then
\begin{enumerate}
\item $TS$ is bounded and $\|TS\|\leq\|T\|\|S\|$;
\item $(TS)^{*}=S^{*}T^{*}$.
\end{enumerate}
\end{prop}

\begin{proof}\hfill

\hfill

1. Let $x\in\mathcal{H}$ such that $\|x\|=1$. Then
\begin{align*}
\|TSx\| & \leq\|T\|\|Sx\|\\
 & \leq\|T\|\|S\|\|x\|\\
 & =\|T\|\|S\|.
\end{align*}
Thus $TS$ is bounded and $\|TS\|\leq\|T\|\|S\|$.

\hfill

2. Let $y\in\mathcal{H}$. Then for all $x\in\mathcal{H}$, we have
\begin{align*}
\langle x,(TS)^{*}y\rangle & =\langle TSx,y\rangle\\
 & =\langle Sx,T^{*}y\rangle\\
 & =\langle x,S^{*}T^{*}y\rangle.
\end{align*}
In particular, this implies $(TS)^{*}y=S^{*}T^{*}y$ for all $y\in\mathcal{H}$,
which implies $(TS)^{*}=S^{*}T^{*}$. \end{proof}

\subsection*{Problem 3}

\begin{prop}\label{propnormuv} Let $u,v\in\mathcal{H}$ be fixed
vectors. 
\begin{enumerate}
\item The operator $T\colon\mathcal{H}\to\mathcal{H}$ defined by
\[
Tx=\langle x,u\rangle v
\]
for all $x\in\mathcal{H}$ is bounded. Moreover, we have $\|T\|=\|u\|\|v\|$.
\item The adjoint of $T$ is given by
\[
T^{*}y=\langle y,v\rangle u
\]
for all $y\in\mathcal{H}$. 
\end{enumerate}
\end{prop}

\begin{proof}\hfill

\hfill

1. Let $x\in\mathcal{H}$. Then
\begin{align*}
\|Tx\| & =\|\langle x,u\rangle v\|\\
 & =|\langle x,u\rangle|\|v\|\\
 & \leq\|x\|\|u\|\|v\|,
\end{align*}
where we used Cauchy-Schwarz to get from the second to the third line.
This implies $\|T\|\leq\|u\|\|v\|$. We have equality at the Cauchy-Schwarz
step if and only if $x=\lambda u$ for some $\lambda\in\mathbb{C}$.
In particular, setting $x=u/\|u\|$ gives us $\|T\|=\|u\|\|v\|$. 

\hfill

2. Let $y\in\mathcal{H}$. Then
\begin{align*}
\langle x,T^{*}y\rangle & =\langle Tx,y\rangle\\
 & =\langle\langle x,u\rangle v,y\rangle\\
 & =\langle x,u\rangle\langle v,y\rangle\\
 & =\langle x,\overline{\langle v,y\rangle}u\rangle\\
 & =\langle x,\langle y,v\rangle u\rangle
\end{align*}
for all $x\in\mathcal{H}$. This implies $T^{*}y=\langle y,v\rangle u$
for all $y\in\mathcal{H}$. \end{proof}

\subsection*{Problem 4}

\begin{cor}\label{cor} Let $T\colon\ell^{2}(\mathbb{N})\to\ell^{2}(\mathbb{N})$
be operator defined by
\[
T(x)_{n}=\sum_{m=1}^{\infty}\frac{x_{m}}{2^{n}3^{m}},
\]
for all $x=(x_{m})\in\ell^{2}(\mathbb{N})$, where $T(x)_{n}$ denotes
the $n$-th coordinate of $T(x)\in\ell^{2}(\mathbb{N})$. Then $T$
is bounded with
\[
\|T\|=\sqrt{\frac{1}{24}}.
\]
The adjoint of $T$ is given by
\[
T^{*}(y)_{n}=\sum_{m=1}^{\infty}\frac{y_{m}}{2^{m}3^{n}},
\]
for all $y\in\ell^{2}(\mathbb{N})$. \end{cor}

\begin{proof} Set $u=(1/3^{m})$ and $v=(1/2^{n})$. Then
\begin{align*}
T(x)_{n} & =\sum_{m=1}^{\infty}\frac{x_{m}}{2^{n}3^{m}}\\
 & =\langle x,u\rangle\frac{1}{2^{n}}\\
 & =\langle x,u\rangle v_{n}
\end{align*}
for all $x\in\mathcal{H}$. Thus $Tx=\langle x,u\rangle v$ for all
$x\in\mathcal{H}$. Therefore we can apply Proposition~(\ref{propnormuv})
and obtain
\begin{align*}
\|T\| & =\|u\|\|v\|\\
 & =\sqrt{\sum_{n=1}^{\infty}9^{-n}}\sqrt{\sum_{n=1}^{\infty}4^{-n}}\\
 & =\sqrt{\left(\frac{1}{1-\frac{1}{9}}-1\right)\left(\frac{1}{1-\frac{1}{4}}-1\right)}\\
 & =\sqrt{\frac{1}{24}}.
\end{align*}
The adjoint of $T$ is given by
\begin{align*}
T^{*}(y)_{n} & =\langle y,v\rangle u_{n}\\
 & =\sum_{m=1}^{\infty}\frac{y_{m}}{2^{m}3^{n}}
\end{align*}
for all $y\in\mathcal{H}$. \end{proof}

\subsection*{Problem 5}

\begin{prop}\label{prop} Let $T\colon\mathcal{H}\to\mathcal{H}$
be a bounded operator. Then
\begin{enumerate}
\item $\|T^{*}T\|=\|T\|^{2}$;
\item $\text{Ker}(T^{*}T)=\text{Ker}(T)$.
\end{enumerate}
\end{prop}

\begin{proof}\hfill 

\hfill

1. First note that Proposition~(\ref{propcomp}) implies $\|T^{*}T\|\leq\|T^{*}\|\|T\|=\|T\|^{2}$.
For the reverse inequality, let $x\in\mathcal{H}$ such that $\|x\|=1$.
Then
\begin{align*}
\|Tx\|^{2} & =\langle Tx,Tx\rangle\\
 & =\langle x,T^{*}Tx\rangle\\
 & \leq\|x\|\|T^{*}Tx\|\\
 & =\|T^{*}Tx\|,
\end{align*}
where we used Cauchy-Schwarz to get from the second line to the third
line. In particular, this implies
\begin{align*}
\|T\|^{2} & =\sup\{\|Tx\|^{2}\mid\|x\|\leq1\}\\
 & \leq\sup\{\|T^{*}Tx\|\mid\|x\|\leq1\}\\
 & =\|T^{*}T\|,
\end{align*}
where the first line is justifed in the Appendix.

\hfill

2. Let $x\in\text{Ker}(T)$. Then
\begin{align*}
T^{*}Tx & =T^{*}(Tx)\\
 & =T^{*}(0)\\
 & =0
\end{align*}
implies $x\in\text{Ker}(T^{*}T)$. Thus $\text{Ker}(T)\subseteq\text{Ker}(T^{*}T)$. 

~~~For the reverse inclusion, let $x\in\text{Ker}(T^{*}T)$. Then
\begin{align*}
\langle Tx,Tx\rangle & =\langle x,T^{*}Tx\rangle\\
 & =\langle x,0\rangle\\
 & =0
\end{align*}
implies $Tx=0$ (by positive-definiteness of inner-product) which
implies $x\in\text{Ker}(T)$. Therefore $\text{Ker}(T)\supseteq\text{Ker}(T^{*}T)$.
\end{proof}

\subsection*{Problem 6}

\begin{prop}\label{prop} Let $T\colon\mathcal{H}\to\mathcal{H}$
be a bounded operator. Then
\begin{enumerate}
\item $\text{Ker}(T^{*})=\text{Im}(T)^{\perp}$;
\item $\text{Ker}(T)^{\perp}=\overline{\text{Im}(T^{*})}$.
\end{enumerate}
\end{prop}

\begin{proof}\hfill

\hfill

1. Let $x\in\text{Ker}(T^{*})$. Then
\begin{align*}
\langle Ty,x\rangle & =\langle y,T^{*}x\rangle\\
 & =\langle y,0\rangle\\
 & =0
\end{align*}
for all $Ty\in\text{Im}(T)$. This implies $x\in\text{Im}(T)^{\perp}$
and so $\text{Ker}(T^{*})\subseteq\text{Im}(T)^{\perp}$. 

~~~For the reverse inclusion, let $x\in\text{Im}(T)^{\perp}$.
Then
\begin{align*}
0 & =\langle x,TT^{*}x\rangle\\
 & =\langle T^{*}x,T^{*}x\rangle
\end{align*}
implies $T^{*}x=0$ (by positive-definiteness of inner-product) which
implies $x\in\text{Ker}(T^{*})$.

\hfill

2. Let us first show that $\text{Ker}(T)^{\perp}$ contains $\text{Im}(T^{*})$.
Let $T^{*}y\in\text{Im}(T^{*})$. Then for all $x\in\text{Ker}(T)$,
we have
\begin{align*}
\langle x,T^{*}y\rangle & =\langle Tx,y\rangle\\
 & =\langle0,y\rangle\\
 & =0.
\end{align*}
In particular, this implies $\overline{\text{Im}(T^{*})}\subseteq\text{Ker}(T)^{\perp}$
(as $\text{Ker}(T)^{\perp}$ is a closed subspace which contains $\text{Im}(T^{*})$). 

~~~For the reverse inclusion, we have
\begin{align*}
\text{Ker}(T)^{\perp} & =\text{Ker}((T^{*})^{*})^{\perp}\\
 & =(\text{Im}(T^{*})^{\perp})^{\perp}\\
 & =(\overline{\text{Im}(T^{*})}{}^{\perp})^{\perp}\\
 & =\overline{\text{Im}(T^{*})},
\end{align*}
where we used part 1 of this proposition to get from the first line
to the second line. \end{proof}

\subsection*{Problem 7}

\begin{defn}\label{defn} An \textbf{isometry} between normed vector
spaces $\mathcal{V}_{1}$ and $\mathcal{V}_{2}$ is an operator $T\colon\mathcal{V}_{1}\to\mathcal{V}_{2}$
such that
\[
\|Tx-Ty\|=\|x-y\|
\]
for all $x,y\in\mathcal{V}$. \end{defn}

\begin{prop}\label{prop} Let $\mathcal{V}_{1}$ and $\mathcal{V}_{2}$
be inner-product spaces and let $T\colon\mathcal{V}_{1}\to\mathcal{V}_{2}$
be an operator. Then $T$ is an isometry (where $\mathcal{V}_{1}$
and $\mathcal{V}_{2}$ are viewed as the induced normed vector spaces
with respect to their inner-products) if and only if
\begin{equation}
\langle x,y\rangle=\langle Tx,Ty\rangle\label{isometry}
\end{equation}
for all $x,y\in\mathcal{V}_{1}$. \end{prop}

\begin{proof} Suppose (\ref{isometry}) holds for all $x,y\in\mathcal{V}_{1}$.
Then
\begin{align*}
\|Tx-Ty\| & =\sqrt{\langle Tx-Ty,Tx-Ty\rangle}\\
 & =\sqrt{\langle Tx,Tx\rangle-\langle Tx,Ty\rangle-\langle Ty,Tx\rangle+\langle Ty,Ty\rangle}\\
 & =\sqrt{\langle x,x\rangle-\langle x,y\rangle-\langle y,x\rangle+\langle y,y\rangle}\\
 & =\sqrt{\langle x-y,x-y\rangle}\\
 & =\|x-y\|.
\end{align*}
for all $x,y\in\mathcal{V}_{1}$. Thus $T$ is an isometry. 

~~~Conversely, suppose $T$ is an isometry and let $x,y\in\mathcal{V}_{1}$.
Then
\begin{align*}
\|x\|^{2}-2\text{Re}(\langle x,y\rangle)+\|y\|^{2} & =\langle x-y,x-y\rangle\\
 & =\langle Tx-Ty,Tx-Ty\rangle\\
 & =\|Tx\|^{2}-2\text{Re}(\langle Tx,Ty\rangle)+\|Ty\|^{2}\\
 & =\|x\|^{2}-2\text{Re}(\langle Tx,Ty\rangle)+\|y\|^{2}
\end{align*}
implies $\text{Re}(\langle x,y\rangle)=\text{Re}(\langle Tx,Ty\rangle)$
for all $x,y\in\mathcal{V}_{1}$. Note that this also implies
\begin{align*}
\text{Im}(\langle x,y\rangle) & =-\text{Re}(i\langle x,y\rangle)\\
 & =-\text{Re}(\langle ix,y\rangle)\\
 & =-\text{Re}(\langle T(ix),Ty\rangle)\\
 & =-\text{Re}(i\langle Tx,Ty\rangle)\\
 & =\text{Im}(\langle Tx,Ty\rangle)
\end{align*}
for all $x,y\in\mathcal{V}_{1}$. Thus we have (\ref{isometry}) for
all $x,y\in\mathcal{V}_{1}$. \end{proof}

\begin{prop}\label{prop} Let $T\colon\mathcal{H}\to\mathcal{H}$
be a bounded operator. Then
\begin{enumerate}
\item $T$ is an isometry if and only if $T^{*}T=1_{\mathcal{H}}$.
\item There exists isometries $T$ such that $TT^{*}\neq1_{\mathcal{H}}$.
\end{enumerate}
\end{prop}

\begin{proof}\hfill

\hfill

1. Suppose $T$ is an isometry. Then for all $y\in\mathcal{H}$, we
have
\begin{align*}
\langle x,1_{\mathcal{H}}y\rangle & =\langle x,y\rangle\\
 & =\langle Tx,Ty\rangle\\
 & =\langle x,T^{*}Ty\rangle
\end{align*}
for all $x\in\mathcal{H}$. In particular, this implies $T^{*}Ty=1_{\mathcal{H}}y$
for all $y\in\mathcal{H}$, which implies $T^{*}T=1_{\mathcal{H}}$. 

~~~Conversely, suppose $T^{*}T=1_{\mathcal{H}}$. Then
\begin{align*}
\langle Tx,Ty\rangle & =\langle x,T^{*}Ty\rangle\\
 & =\langle x,1_{\mathcal{H}}y\rangle\\
 & =\langle x,y\rangle
\end{align*}
for all $x,y\in\mathcal{H}$. This implies $T$ is an isometry. 

\hfill

2. Consider the shift operator $S\colon\ell^{2}(\mathbb{N})\to\ell^{2}(\mathbb{N})$,
given by
\[
S(x_{n})=(x_{n-1})
\]
for all $(x_{n})\in\ell^{2}(\mathbb{N})$, where $x_{0}=0$. In class,
it was shown that
\[
S^{*}(x_{n})=(x_{n+1})
\]
for all $(x_{n})\in\ell^{2}(\mathbb{N})$. Thus, whenever $x_{1}\neq0$,
we have
\begin{align*}
SS^{*}(x_{n}) & =SS^{*}(x_{1},x_{2},\dots)\\
 & =S(x_{2},x_{3},\dots)\\
 & =(0,x_{2},x_{3},\dots)\\
 & \neq(x_{n}).
\end{align*}
On the other hand, $S$ is an isometry. Indeed, let $(x_{n}),(y_{n})\in\ell^{2}(\mathbb{N})$.
Then
\begin{align*}
\langle S(x_{n}),S(y_{n})\rangle & =\langle(x_{n-1}),(y_{n-1})\rangle\\
 & =\sum_{n=1}^{\infty}x_{n-1}\overline{y}_{n-1}\\
 & =\sum_{m=0}^{\infty}x_{m}\overline{y}_{m}\\
 & =x_{0}y_{0}+\sum_{m=1}^{\infty}x_{m}\overline{y}_{m}\\
 & =\sum_{m=1}^{\infty}x_{m}\overline{y}_{m}\\
 & =\langle(x_{n}),(y_{n})\rangle.
\end{align*}
 

\end{proof}

\section*{Appendix}

\begin{prop}\label{prop} Let $T\colon\mathcal{U}\to\mathcal{V}$
be a bounded linear operator. Then
\[
\|T\|^{2}=\sup\{\|Tx\|^{2}\mid\|x\|\leq1\}
\]

\end{prop}

\begin{proof} For any $x\in\mathcal{U}$ such that $\|x\|\leq1$,
we have $\|Tx\|^{2}\leq\|T\|^{2}$. Thus 
\begin{equation}
\|T\|^{2}\geq\sup\{\|Tx\|^{2}\mid\|x\|\leq1\}.\label{eq:strict}
\end{equation}
To show the reverse inequality, we assume (for a contradiction) that
(\ref{eq:strict}) is a strictly inequality. Choose $\delta>0$ such
that 
\[
\|T\|^{2}-\delta>\sup\{\|Tx\|^{2}\mid\|x\|\leq1\}.
\]
Now let $\varepsilon=\delta/2\|T\|$, and choose $x\in\mathcal{U}$
such that $\|x\|\leq1$ and such that
\[
\|T\|-\varepsilon<\|Tx\|.
\]
Then 
\begin{align*}
\|Tx\|^{2} & >(\|T\|-\varepsilon)^{2}\\
 & =\|T\|^{2}-2\varepsilon\|T\|+\varepsilon^{2}\\
 & \geq\|T\|^{2}-2\varepsilon\|T\|\\
 & =\|T\|^{2}-\delta
\end{align*}
gives us a contradiction. \end{proof}
\end{document}
