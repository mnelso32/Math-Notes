%% LyX 2.3.3 created this file.  For more info, see http://www.lyx.org/.
%% Do not edit unless you really know what you are doing.
\documentclass[12pt,english]{article}
\usepackage[osf]{mathpazo}
\renewcommand{\sfdefault}{lmss}
\renewcommand{\ttdefault}{lmtt}
\usepackage[T1]{fontenc}
\usepackage[latin9]{inputenc}
\usepackage[paperwidth=30cm,paperheight=35cm]{geometry}
\geometry{verbose,tmargin=2cm,bmargin=2cm}
\setlength{\parindent}{0bp}
\usepackage{mathrsfs}
\usepackage{amsmath}
\usepackage{amssymb}

\makeatletter
\@ifundefined{date}{}{\date{}}
%%%%%%%%%%%%%%%%%%%%%%%%%%%%%% User specified LaTeX commands.
\usepackage{tikz}
\usetikzlibrary{matrix,arrows,decorations.pathmorphing}
\usetikzlibrary{shapes.geometric}
\usepackage{tikz-cd}
\usepackage{amsthm}
\usepackage{xparse,etoolbox}

\theoremstyle{plain}
\newtheorem{theorem}{Theorem}[section]
\newtheorem{lemma}[theorem]{Lemma}
\newtheorem{prop}{Proposition}[section]
\newtheorem*{cor}{Corollary}
\theoremstyle{definition}
\newtheorem{defn}{Definition}[section]
\newtheorem{ex}{Exercise} 
\newtheorem{example}{Example}[section]
\theoremstyle{remark}
\newtheorem*{rem}{Remark}
\newtheorem*{note}{Note}
\newtheorem{case}{Case}
\usepackage{graphicx}
\usepackage{amssymb}
\usepackage{tikz-cd}
\usetikzlibrary{calc,arrows,decorations.pathreplacing}
\tikzset{mydot/.style={circle,fill,inner sep=1.5pt},
commutative diagrams/.cd,
  arrow style=tikz,
  diagrams={>=latex},
}

\usepackage{babel}
\usepackage{hyperref}
\hypersetup{
    colorlinks,
    citecolor=blue,
    filecolor=blue,
    linkcolor=blue,
    urlcolor=blue
}
\usepackage{pgfplots}
\usetikzlibrary{decorations.markings}
\pgfplotsset{compat=1.9}


\newcommand{\blocktheorem}[1]{%
  \csletcs{old#1}{#1}% Store \begin
  \csletcs{endold#1}{end#1}% Store \end
  \RenewDocumentEnvironment{#1}{o}
    {\par\addvspace{1.5ex}
     \noindent\begin{minipage}{\textwidth}
     \IfNoValueTF{##1}
       {\csuse{old#1}}
       {\csuse{old#1}[##1]}}
    {\csuse{endold#1}
     \end{minipage}
     \par\addvspace{1.5ex}}
}

\raggedbottom

\blocktheorem{theorem}% Make theo into a block
\blocktheorem{defn}% Make defi into a block
\blocktheorem{lemma}% Make lem into a block
\blocktheorem{rem}% Make rem into a block
\blocktheorem{cor}% Make col into a block
\blocktheorem{prop}% Make prop into a block


\usepackage[bottom]{footmisc}

\makeatother

\usepackage{babel}
\begin{document}
\title{Linear Analysis Homework 2}
\author{Michael Nelson}

\maketitle
Throughout this homework, let $\mathscr{V}$ be an inner-product space
over $\mathbb{C}$. If $x\in\mathscr{V}$ and $r>0$, then we define
\[
B_{r}(x):=\{y\in\mathscr{V}\mid\|y-x\|<r\}
\]
to be the \textbf{open ball centered at $x$ and of radius $r$}.
We also define 
\[
B_{r}[x]:=\{y\in\mathscr{V}\mid\|y-x\|\leq r\}
\]
to be the \textbf{closed ball centered at $x$ and of radius $r$}. 

\subsection*{Problem 1}

\begin{prop}\label{prop} Let $a\in\mathscr{V}$ and $r>0$. Then
\[
B_{r}(a)=a+rB_{1}(0).
\]
\end{prop}

\begin{proof} We prove this in two steps. 

\hfill

\textbf{Step 1: }We show $B_{r}(a)=a+B_{r}(0)$: Let $x\in B_{r}(a)$,
so $\|x-a\|<r$. This implies $x-a\in B_{r}(0)$. Thus
\begin{align*}
x & =a+(x-a)\\
 & \in a+B_{r}(0).
\end{align*}
Therefore $B_{r}(a)\subseteq a+B_{r}(0)$. 

~~~Conversely, let $a+y\in a+B_{r}(0)$ where $y\in B_{r}(0)$,
so $\|y\|<r$. This implies $\|(a+y)-a\|<r$. In other words, $a+y\in B_{r}(a)$.
Therefore $a+B_{r}(0)\subseteq B_{r}(a)$. 

\hfill

\textbf{Step 2: }We show $B_{r}(0)=rB_{1}(0)$: Let $x\in B_{r}(0)$,
so $\|x\|<r$. Then since $r>0$, we have
\begin{align*}
1 & >(1/r)\|x\|\\
 & =\|x/r\|.
\end{align*}
In other words, $x/r\in B_{1}(0)$. Thus
\begin{align*}
x & =r(x/r)\\
 & \in rB_{1}(0).
\end{align*}
Therefore $B_{r}(0)\subseteq rB_{1}(0)$. 

~~~Conversely, let $ry\in rB_{1}(0)$ where $y\in B_{1}(0)$, so
$\|y\|<1$. Then since $r>0$, we have
\begin{align*}
\|ry\| & =r\|y\|\\
 & <1.
\end{align*}
In other words, $ry\in B_{1}(0)$. Therefore $rB_{1}(0)\subseteq B_{r}(0)$.
\end{proof}

\subsection*{Problem 2}

\begin{lemma}\label{lemmasequence1overn} Let $x\in\mathscr{V}$ and
for each $n\in\mathbb{N}$ let $x_{n}\in\mathscr{V}$ such that $\|x_{n}-x\|<1/n$.
Then $x_{n}\to x$. \end{lemma}

\begin{proof} Let $\varepsilon>0$. Choose $N\in\mathbb{N}$ such
that $1/N<\varepsilon$. Then $n\geq N$ implies
\begin{align*}
\|x_{n}-x\| & <1/n\\
 & \leq1/N\\
 & <\varepsilon.
\end{align*}
\end{proof}

\begin{prop}\label{prop} Let $a\in\mathscr{V}$ and $r>0$. Then
\[
\overline{B_{r}(a)}=B_{r}[a].
\]
\end{prop}

\begin{proof} Let $x\in\overline{B_{r}(a)}$. Choose a sequence $(x_{n})$
of elements in $B_{r}(a)$ such that $x_{n}\to x$. Let $\varepsilon>0$
and choose $N\in\mathbb{N}$ such that $\|x_{N}-x\|<\varepsilon$.
Then
\begin{align*}
\|x-a\| & =\|x-x_{N}+x_{N}-a\|\\
 & \leq\|x-x_{N}\|+\|x_{N}-a\|\\
 & <\varepsilon+r.
\end{align*}
Thus $\|x-a\|<r+\varepsilon$ for all $\varepsilon>0$. This implies
$\|x-a\|\leq r$, or in other words, $x\in B_{r}[a]$. Thus $\overline{B_{r}(a)}\subseteq B_{r}[a]$. 

~~~Conversely, let $x\in B_{r}[a]$ and let $n\in\mathbb{N}$.
We first observe that for each $t\in(0,1)$, we have
\begin{align*}
\|(x+t(a-x))-a\| & =\|(1-t)x-(1-t)a\|\\
 & =(1-t)\|x-a\|\\
 & <r.
\end{align*}
Thus $x+t(a-x)\in B_{r}(a)$ for all $t\in(0,1)$. Now let $n\in\mathbb{N}$.
Choose $t_{n}\in(0,1)$ such that $t_{n}<\|x-a\|/n$. Then
\begin{align*}
\|(x+t_{n}(a-x))-x\| & =\|t_{n}(x-a)\|\\
 & =t_{n}\|x-a\|\\
 & <1/n.
\end{align*}
Thus $(x+t_{n}(a-x))$ is a sequence of elements of elements in $B_{r}(a)$
such that $x+t_{n}(a-x)\to x$ (by Lemma~(\ref{lemmasequence1overn})),
hence $x\in\overline{B_{r}(a)}$. Thus $B_{r}[a]\subseteq\overline{B_{r}(a)}$.
\end{proof}

\subsection*{Problem 3a}

\begin{lemma}\label{lemma} Let $A\subseteq\mathscr{V}$. \end{lemma}

\begin{prop}\label{prop} Let $A\subseteq\mathscr{V}$ and let $C_{1},C_{2}\subseteq\mathscr{V}$
such that $C_{1}$ and $C_{2}$ are closed. Then 
\begin{enumerate}
\item $\overline{A}$ is a closed set.
\item $\overline{A}$ is the smallest closed set that contains $A$, i.e.,
for any closed set $B$ such that $A\subseteq B$ we have $\overline{A}\subseteq B$.
In particular, $\overline{A}=A$ if and only if $A$ is closed. 
\item The union of $C_{1}$ and $C_{2}$ is closed.
\item The intersection of $C_{1}$ and $C_{2}$ is closed. 
\item An infinite union of closed sets may not be closed. 
\end{enumerate}
\end{prop}

\begin{proof} \hfill
\begin{enumerate}
\item We will show that $\overline{A}$ is closed by showing that $\mathscr{V}\backslash\overline{A}$
is open. To show that $\mathscr{V}\backslash\overline{A}$ is open,
it suffices to show that for each $x\in\mathscr{V}\backslash\overline{A}$
there exists an open neighborhood of $x$ which is contained in $\mathscr{V}\backslash\overline{A}$.
Assume (for a contradiction) that $\mathscr{V}\backslash\overline{A}$
is not open. Choose $x\in\mathscr{V}\backslash\overline{A}$ such
that every open neighborhood of $x$ meets $\overline{A}$. In particular,
for each $n\in\mathbb{N}$, there exists $x_{n}\in B_{1/n}(x)\cap\overline{A}$.
Choose such $x_{n}$ for all $n\in\mathbb{N}$. Then by Lemma~(\ref{lemmasequence1overn}),
we must have $x_{n}\to x$, and hence $x\in\overline{\overline{A}}=\overline{A}$.
This is a contradiction.
\item Let $B$ be any closed set which contains $A$. Suppose $x\in\overline{A}$.
Choose a sequence $(x_{n})$ of elements in $A$ such that $x_{n}\to x$.
Assume (for a contradiction) that $x\in\mathscr{V}\backslash B$.
Choose $\varepsilon>0$ such that $B_{\varepsilon}(x)\cap B=\emptyset$
(we can do this since $\mathscr{V}\backslash B$ is open). But then
the sequence $(x_{n})$ of elements in $B$ cannot converge to $x$
since $x_{n}\notin B_{\varepsilon}(x)$ for all $n\in\mathbb{N}$.
This is a contradiction. For the last statement. If $A$ is closed,
then since $\overline{A}$ is the smallest\emph{ }closed set containing
$A$, we must have $A=\overline{A}$. And if $A=\overline{A}$, then
since $\overline{A}$ is the smallest\emph{ }closed set containing
$A$, the set $A$ itself must be closed. 
\item Combining 2 with an identity we proved in class, we have
\begin{align*}
C_{1}\cup C_{2} & =\overline{C}_{1}\cup\overline{C}_{2}\\
 & =\overline{C_{1}\cup C_{2}}.
\end{align*}
Therefore $C_{1}\cup C_{2}$ is closed. 
\item Combining 2 with a couple identities that we proved in class, we have
\begin{align*}
\overline{C_{1}\cap C_{2}} & \supseteq C_{1}\cap C_{2}\\
 & =\overline{C}_{1}\cap\overline{C}_{2}\\
 & \supseteq\overline{C_{1}\cap C_{2}}.
\end{align*}
Therefore $C_{1}\cap C_{2}$ is closed. 
\item Consider $\mathscr{V}=\mathbb{R}$ and $C_{n}=[0,1-1/n]$ for all
$n\in\mathbb{N}$. Then $\bigcup_{n=1}^{\infty}C_{n}=[0,1)$, which
is not closed in $\mathbb{R}$. 
\end{enumerate}
\end{proof}

\subsection*{Problem 4}

\begin{prop}\label{prop} Let $E\subseteq\mathscr{V}$ and let $x,y\in\mathscr{V}$.
Then 
\begin{enumerate}
\item $d(x,E)=0$ if and only if $x\in\overline{E}$;
\item $|d(x,E)-d(y,E)|\leq\|x-y\|$.
\end{enumerate}
\end{prop}

\begin{proof}\hfill
\begin{enumerate}
\item First suppose that $d(x,E)=0$. For each $n\in\mathbb{N}$, choose
$x_{n}\in E$ such that $\|x_{n}-x\|<1/n$ (if we couldn't find such
an $x_{n}$, then $0$ would not be the infinum). Now we apply Lemma~(\ref{lemmasequence1overn})
to find that $(x_{n})$ is a sequence of elements in $E$ such that
$x_{n}\to x$. Therefore $x\in\overline{E}$. Conversely, suppose
that $x\in\overline{E}$. Choose a sequence $(x_{n})$ of elements
in $E$ such that $x_{n}\to x$. Then we have 
\[
0\leq d(x,E)<\|x_{n}-x\|
\]
for all $n\in\mathbb{N}$. This implies $d(x,E)=0$. 
\item Without loss of generality, we may assume that $d(x,E)\geq d(y,E)$.
Thus we are trying to show that $d(x,E)\leq\|x-y\|+d(y,E)$. Choose
$y_{n}\in E$ such that $\|y_{n}-y\|<d(y,E)+1/n$ for all $n\in\mathbb{N}$.
Then 
\begin{align*}
d(x,E) & \leq\|x-y_{n}\|\\
 & =\|x-y+y-y_{n}\|\\
 & \leq\|x-y\|+\|y-y_{n}\|\\
 & <\|x-y\|+d(y,E)+1/n.
\end{align*}
Taking $n\to\infty$ gives us our desired result. 
\end{enumerate}
\end{proof}

\subsection*{Problem 5}

\begin{prop}\label{prop} Let $\mathscr{H}$ be a Hilbert space of
$\mathbb{C}$, let $\mathscr{K}$ be a closed subspace of $\mathscr{H}$,
let $x,y\in\mathscr{H}$, and let $\lambda\in\mathbb{C}$. Then 
\begin{enumerate}
\item $d(\lambda x,\mathscr{K})=|\lambda|d(x,\mathscr{K})$;
\item $d(x+y,\mathscr{K})\leq d(x,\mathscr{K})+d(y,\mathscr{K})$.
\end{enumerate}
\end{prop}

\begin{proof} \hfill
\begin{enumerate}
\item Choose a sequence $(y_{n})$ of elements in $\mathcal{A}$ such that
\[
\|x-y_{n}\|<d(x,\mathcal{A})-1/n
\]
for all $n\in\mathbb{N}$. Then
\begin{align*}
\|\lambda x-\lambda y_{n}\| & =|\lambda|\|x-y_{n}\|\\
 & <|\lambda|d(x,\mathcal{A})-|\lambda|/n
\end{align*}
\[
|\lambda|d(x,\mathcal{A})\leq|\lambda|\|x-y_{n}\|<d(x,\mathcal{A})-1/n
\]
\item Let $a$ be the unique element in $\mathscr{K}$ such that $d(x,\mathscr{K})=\|x-a\|$.
Then $\lambda a\in\mathscr{K}$, and so
\begin{align*}
|\lambda|d(x,\mathscr{K}) & =|\lambda|\|x-a\|\\
 & =\|\lambda x-\lambda a\|\\
 & \geq d(\lambda x,\mathscr{K}).
\end{align*}
Conversely, let $b$ be the unique element in $\mathscr{K}$ such
that $d(\lambda x,\mathscr{K})=\|x-b\|$. Then $b/\lambda\in\mathscr{K}$,
and so
\begin{align*}
d(\lambda x,\mathscr{K}) & =\|\lambda x-b\|\\
 & =|\lambda|\|x-b/\lambda\|\\
 & \geq|\lambda|d(x,\mathscr{K}).
\end{align*}
\item Let $a$ be the unique element in $\mathscr{K}$ such that $d(x,\mathscr{K})=\|x-a\|$
and let $b$ be the unique element in $\mathscr{K}$ such that $d(y,\mathscr{K})=\|y-b\|$.
Then $a+b\in\mathscr{K}$, and so 
\begin{align*}
d(x+y,\mathscr{K}) & \leq\|x+y-(a+b)\|\\
 & =\|x-a\|+\|y-b\|\\
 & =d(x,\mathscr{K})+d(y,\mathscr{K}).
\end{align*}
\end{enumerate}
\end{proof}
\end{document}
