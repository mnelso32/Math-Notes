%% LyX 2.2.3 created this file.  For more info, see http://www.lyx.org/.
%% Do not edit unless you really know what you are doing.
\documentclass[english]{article}
\usepackage[osf]{mathpazo}
\renewcommand{\sfdefault}{lmss}
\renewcommand{\ttdefault}{lmtt}
\usepackage[T1]{fontenc}
\usepackage[latin9]{inputenc}
\usepackage[paperwidth=30cm,paperheight=35cm]{geometry}
\geometry{verbose,tmargin=3cm,bmargin=3cm}
\setlength{\parindent}{0bp}
\usepackage{amsmath}
\usepackage{amssymb}

\makeatletter
%%%%%%%%%%%%%%%%%%%%%%%%%%%%%% User specified LaTeX commands.
\usepackage{tikz}
\usetikzlibrary{matrix,arrows,decorations.pathmorphing}
\usetikzlibrary{shapes.geometric}
\usepackage{tikz-cd}
\usepackage{amsthm}
\theoremstyle{plain}
\newtheorem{theorem}{Theorem}[section]
\newtheorem{lemma}[theorem]{Lemma}
\newtheorem{prop}{Proposition}[section]
\newtheorem*{cor}{Corollary}
\theoremstyle{definition}
\newtheorem{defn}{Definition}[section]
\newtheorem{ex}{Exercise} 
\newtheorem{example}{Example}[section]
\theoremstyle{remark}
\newtheorem*{rem}{Remark}
\newtheorem*{note}{Note}
\newtheorem{case}{Case}
\usepackage{graphicx}
\usepackage{amssymb}
\usepackage{tikz-cd}
\usetikzlibrary{calc,arrows,decorations.pathreplacing}
\tikzset{mydot/.style={circle,fill,inner sep=1.5pt},
commutative diagrams/.cd,
  arrow style=tikz,
  diagrams={>=latex},
}

\usepackage{babel}
\usepackage{hyperref}
\hypersetup{
    colorlinks,
    citecolor=black,
    filecolor=black,
    linkcolor=black,
    urlcolor=black
}
\usepackage{pgfplots}
\usetikzlibrary{decorations.markings}
\pgfplotsset{compat=1.9}

\makeatother

\usepackage{babel}
\begin{document}

\title{Combinatorics Homework}

\maketitle
$(1.1.2)$: No: If we give the board a checkerboard coloring, i.e.
each $1\times1$ square is colored black or white and no two adjacent
$1\times1$ squares are colored the same, then every tiling of the
board must cover the same amount of $1\times1$ black squares as there
are $1\times1$ white squares. But no matter which checkerboard coloring
we choose, there will always be more $1\times1$ squares of one color
than $1\times1$ squares of the other color, since any diagonally
oppositive $1\times1$ squares have the same color. 

\hfill

$(1.1.6):$ Yes: Without loss of generality, suppose we remove the
squares at positions $(1,1)$ and $(m,n)$. The board is made up of
two $m-1\times1$ boards and a $n-2\times m$ board. Each of these
boards can be tiled since $n-2$ and $m-1$ are even. Putting everything
together gives us a tiling.

\hfill

$(1.2.1):$ A positive factor of $2\cdot3^{4}\cdot7^{3}\cdot11^{2}\cdot47^{5}$
has the form $2^{e_{2}}\cdot3^{e_{3}}\cdot7^{e_{7}}\cdot11^{e_{11}}\cdot47^{e_{47}}$
where $e_{2}\in\{0,1\}$, $e_{3}\in\{0,1,2,3,4\}$, $e_{7}\in\{0,1,2,3\}$,
$e_{11}\in\{0,1,2\}$, $e_{47}\in\{0,1,2,3,4,5\}$. There are $2$
choices for $e_{2}$, $5$ choices for $e_{3}$, $4$ choices for
$e_{7}$, $3$ choices for $e_{11}$, $6$ choices for $e_{47}$.
Since these choices are independent of each other, there are $2\cdot5\cdot4\cdot3\cdot6$
positive factors of $2\cdot3^{4}\cdot7^{3}\cdot11^{2}\cdot47^{5}$.
By the same reasoning, the number of positive factors of $n=p_{1}^{e_{1}}\cdots p_{k}^{e_{k}}$
is $(e_{1}+1)\cdots(e_{k}+1)$.

\hfill

$(1.2.4):$ Without loss of generality, assume person $X$ is at the
top of the table as illustrated in the picture above

\begin{center}\begin{tikzpicture}

\node[circle, fill=black, inner sep=1.5pt, label=above:$X$] (a) at (2,4.2) {$$};

\draw (2,2) circle (2cm);

\end{tikzpicture} \end{center}Every seating arrangement gives an ordering to each person: the person
who sits directly next to $X$ in the counter-clockwise direction
is the $1$st person. The person who sits directly next to the $1$st
person in the counter-clockwise direction is the $2$nd person, and
so on. So there are $7!$ ways of seating everyone around the table.
There are $2\cdot6!$ factorial ways of seating everyone around the
table so that $X$ is next to $Y$. The factor $2$ comes from the
choice of $Y$ sitting to the left of $X$ or to the right of $X$.
So there are $7!-2\cdot6!=3600$ different possibilities. 

\hfill

$(1.2.5):$ There are $8^{2}$ choices to place the first rook. Suppose
we place the rook at the position $(i,j)$. Then the next rook cannot
be placed in the $i$'th row or $j$'th column. By deleting the $i$'th
row and $j$'th column, and joining the remaining pieces together
in the obvious way, we get a $7\times7$ board. Thus there are $7^{2}$
choices to place the next rook. Repeating this process, we conclude
that there are $8^{2}\cdot7^{2}\cdot6^{2}\cdot5^{2}\cdot4^{2}\cdot3^{2}\cdot2^{2}\cdot1^{2}=(8!)^{2}$
choices in total. If the board is a $10\times10$ board, then by the
same reasoning there are $10^{2}\cdot9^{2}\cdot8^{2}\cdot7^{2}\cdot6^{2}\cdot5^{2}\cdot4^{2}\cdot3^{2}=\frac{(10!)^{2}}{(2!)^{2}}$
choices in total.


\end{document}
